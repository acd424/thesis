\chapter{Core Related Theoretical Frameworks} This section introduces some especially relevant theories form the wider research field of crime science that underpin the general approach of POP. The routine activity theory \parencite{cohen1979social}and then situational crime prevention \parencite{clarke1997situational} are explored individually to help build the concepts on which POP is based.

\begin{figure}
  \includegraphics[width=\linewidth]{transfer_figs/Slide3.jpeg}
  \caption{Eck's Crime Triangle, reproduced from \cite{eck2003police}}
  \label{fig:triangle}
\end{figure}

\section{Routine Activity Theory}  First introduced by Cohen and Felson in 1979, Routine Activity Theory was proposed as a theory to help explain the increase in crime after WW2. The change that the theory brought about was a shift from thinking about crime purely as a social process to seeing it more as a socio-physical world \parencite{ratchap4}. The focus was on the crime event itself and what conditions were needed for the event to be created. This focus on the crime event has obvious parallels with the focus on the problem in POP, and indeed, the theory has been extended since its first inception to move beyond crime.

In their original article, \parencite{cohen1979social}, Felson and Cohen sought to explain the crime event through the convergence of three principle physical aspects, that is 1) a likely offender, 2) a suitable target and 3) the absence of a capable guardian. These three elements, modified and extended by \parencite{eck2003police} to form the problem triangle in Figure \ref{fig:triangle}, demonstrate the close ties between the two bodies of work. For instance, if a crime or problem opportunity is generated through people’s movements and the types of activities they conduct, then by extension, modifying these activities should also affect the prevalence of crime opportunities.

Therefore, from a POP standpoint, thanks to routine activities theory, there are now at least three broad opportunities to prevent crime. That is, by adapting one of those three physical aspects outlined above, the problem triangle can be broken, and the problem opportunity is lost. This contrasts with the traditional model of policing, which concentrates on a narrow aspect of the offender, namely dissuading him or her through a deterrent effect (police response is likely to catch you) or a removal effect (locking them up prevents their ability to commit crime outside of confinement).


The problem triangle in  Figure \ref{fig:triangle}, with its added outer layer, suggests three broad means with which it can be broken to eradicate the opportunity. The handler has an effect on the potential offender – perhaps their presence physically or emotionally makes the offender less likely to commit an untoward act. The guardian protects the target from would be offenders, and this can vary from a person actively guarding their luggage to unintentional increased footfall in residential areas reducing the opportunity for burglary  \parencite{halford2020crime}. Place managers govern how a place functions \parencite{popchap11}. They may be bar managers, shop designers or teachers. They play an important part in the opportunity structures that arise through the way business is conducted and how the physical environment is set out. Identifying problems and influencing this group of people to change their environment is a good example of a strategy originating from the POP framework.



\section{Situational Crime Prevention} Situation crime prevention (SCP) rests on this claim \say{Reducing Opportunities for specific forms of crime will reduce the overall amounts of crime}\parencite{scpchap13}. Like POP, this theory focusses on the target and the place of crime –   \say{It seeks to forestall the occurrence of crime, rather than to detect and sanction offenders}\parencite{clarke1997situational}. focussing on what can be changed now to have an almost immediate effect on the cause of crimes \parencite{clarke1995situational}. The principles for situational crime prevention are very similar to POP, as explored below:

\begin{enumerate}

\item{\bf{Focus on specific categories of crime.}} Situational crime prevention works best by only attempting to tackle one type of crime at a time, calling for specificity in defining how these crimes are conducted and how the opportunities have been generated \parencite{felson1998opportunity}.If the categories are grouped too widely in the first instance, then common patterns will not be found, and common solutions will be unlikely to work.

\item{\bf{Understand how the crime is committed.}} The focus is on how, not why the crimes were committed – by understanding how they were committed, the mechanism can be interrupted, and the crime can be prevented. Again, any information that is related to how a crime is committed will be useful in its prevention. It is important to note that while some of this information may be found in police reports, they are unlikely to reflect the full range of actions before and after the criminal act.

\item{\bf{Use an Action Research model.}} \say{Action research is a philosophy and methodology of research generally applied in the social sciences. It seeks transformative change through the simultaneous process of taking action and doing research, which are linked together by critical reflection} \footnote{Wikipedia : Action research}. To most practitioners, this probably means doing what you normally do, \emph{observing} the problem, \emph{orientating} to the problem, \emph{deciding} what to do and \emph{acting} on that information, correcting as one proceeds by continually cycling through these stages (the OODA loop as developed by John Boyyd \parencite{modernstrat}). However, it may be useful to highlight the importance of both deeply understanding a problem and acting on that information to combat the crime, rather than using the two strands in isolation.

\item{\bf{Consider a variety of solutions.}} Be open to a whole host of solutions in order to bring around the desired effect. A selection (twenty-five) of generic solutions have been posited as a good starter from which to initially pick and then adapt solutions to implement. Although too many to list here the five groupings of situational crime prevention give a flavour of the spread of solutions that are available. The groupings are as follows:

\begin{enumerate}
\item{\bf{Increase the effort.}} Make crimes harder to commit such as an additional layer of security to overcome e.g. a steering lock on a car.
\item{\bf{Increase the risks.} }Generally increase the chance of being caught, such as through increased surveillance.
\item{\bf{Reduce the rewards.}} Make the crime less attractive. For example, by making stolen goods harder to sell, their value is decreased, and the rewards reduced.
\item{\bf{Reduce the provocation.}} Lessen flash points. An example might be to make bars less crowded to reduced unwelcome interactions.
\item{\bf{Remove excuses.}} Provide obvious information so that ignorance can not be used as an excuse. Erect signs to remind potential offenders of rules in specific areas. 
\end{enumerate}

\end{enumerate}


Situational crime prevention has been criticised on a number of levels for being a superficial technique for reducing crime \parencite{wortley2010critiques}. Some of the more relevant critiques to this research include:

\begin{enumerate}

 \item{\bf{Crime is not reduced only displaced.} }Some critics, especially those who believe that the amount of crime is largely driven by people’s propensity to commit crime and not by situational factors, believe that preventing crime in one area will result in a similar increase in crime in another area – that is, the crime will be displaced rather than prevented. However, a systematic review into this issue \parencite{guerette2009assessing} found that while there are some instances of crime moving to other areas after an intervention, the net benefit was still a reduction in crime. However, they do indicate that controlling in different areas for displacement is difficult, as the displacement may manifest itself temporally, spatially or even to different crimes altogether.\textcite{guerette2009assessing} also do not seem to control for publication bias, which may have a detrimental effect, there being a difference in displacement outcome depending on the overall primary success of the individual studies.
 
 \item{\bf{SCP doesn't work for expressive crimes.}} Although critics accept that high-volume acquisitive crimes can be reduced, they believe that crimes that are more expressive or are \emph{irrational} will not be as easily affected by situational crime prevention techniques. Expressive crimes include domestic violence, sexual offences or those committed in the heat of the moment. However, within the toolkit of SCP, there are efforts to reduce provocation that may prevent manifestation of expressive crime conditions, and there are examples of situational crime prevention projects hosted on the virtual problem orientated policing centre. However, given that \parencite{guerette2009assessing} do not explicitly mention expressive crimes for the studies they used for the systematic review, there is perhaps a practical deficit, if not a theoretical one.
 
 \end{enumerate}


\section{Conclusion} This section has demonstrated that crimes can be prevented from occurring by disrupting the process that create conditions for that crime. The processes centre around the target of the crime, the place that the target is in and the offender. Disrupting the conditions of any these three factors can be enough to prevent the occurrence of a crime, in much the same way as preventing the coming together of the fire triangle elements prevents a fire. Situational crime prevention has been developed to exploit this principal and by doing so has posited five main intervention types that can be used to reduce crime. However, for these interventions to work, they need to be aligned to the crime and its context. POP and the way it takes the theories learnt so far and attempts to distil them into a practical policing framework are explored next in more detail. 


%%%% clarke/goldstien joint papers

%\subsection{Crime Scripts} ``Crime scripts are designed to provide a standardized [sic] systematic and comprehensive understanding of the crime-commission process" \parencite{scriptchap6} Essentially crime scripts are exactly what they appear to be - they are a sequence of events that must be followed in order for a crime to take place. The events can be quite generic, such as say{Target Selection} for a burglary \parencite{scriptchap6}, but this generalisation allows pinch points in the crime process to be identified and actions targeted against specific events, rather than the process as a whole. Thus the benefit of crime scripting, is that the whole process of a crime is understood in more process detail so that crimes or problems with similar events types can be collectively targeted. Crime scripts are intended to reflect all events leading up and after the crime act, thus sometime it is quite difficult to assemble all of the information, as \textcite{scriptchap6} alludes to the best source of information is offenders themselves, and frequently information from the police and victims only offers partial coverage. Despite this crime scripts have shown their utility across a number of crime types and have enabled crime prevention efforts by identifying discrete intervention points that can then be used to disrupt the crime process. The utility for POP is that crime scripts presents an opportunity to identify a specific, perhaps low energy, intervention and then causally interpret the success. In relation to evaluation, comparing crime scripts before and after an intervention can also tease apart the procedural success of the intervention in disrupting the existing process against the ultimate outcome of a reduction in harm. Although the latter is what is wanted understanding all changes made to the problem space is necessary for a complete evaluation.
