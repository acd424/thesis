\chapter{Study 1c: PF2 Burglary MO}

\section{Introduction} The primary aim of this chapter is to examine the potential for replication of the work that has been presented so far. Replication is fundamentally important if work is to be undertaken on a large scale. Police forces in the UK can have different processes and training standards. Therefore, what works in one force area would not necessarily also work in another. The main rationale of this study is to provide further evidence for the purposes of Supporting Objective 2, \say{Evaluate how effective PTMs are with MO data}.
 
In addition to replication, this chapter extends the analytic approach in four respects that reflect the different types of available data. First, the data allowed for an exploration of bias against victims with certain characteristics. Secondly, the data allowed for the completion of an additional classification task, namely an examination of burglaries that involve entering an outbuilding. Third, it was possible to use the models that were developed for the PF1 data to label the PF2 data. This provides insights into the applicability of the models in areas that are covered by different police forces. Fourth, the models were fine-tuned on data from one year and used to label data from a subsequent year. This procedure sheds light on the decay of analytic performance over time.

The main finding of this chapter is that the results from PF1 are largely replicated with PF2, with no significant decline in performance. Accordingly, the key conclusion of this chapter is that PTMs are likely to be applicable at police forces other than the ones that are tested here.

\subsection{Research Questions} This chapter aims to answer three research questions:

\subsubsection{Can the results in Study 1a, the PF1 burglary study, be replicated in a different police force?} Study 1a inquired whether PTMs can be used to classify burglary MO texts in two different scenarios, which have to do with the use of force and the theft of motor vehicles. In each case, the PTMs were finetuned on PF1 data, resulting in appropriate accuracy. In this study, the PTMs are fine-tuned on the same problems but with PF2 data. In addition, it was also possible to build a model for the outbuilding only model. The outbuilding only model was introduced in Study 1a, but it was not completed because the PF1 data did not contain references to that type of burglary. The outbuilding only model is intended to detect whether a burglary targeted only an outbuilding, such as a shed, without the main home of the victim being breached.

\subsubsection{Can models trained with data from one police force be used in another force?} Fine-tuning PTMs on the same task at two different police forces enables the models to be used across areas. It also becomes possible to ascertain whether they are generalisable. If their applicability is indeed broad, then large benefits are likely to result from the dissemination of the models across police forces, which would reduce the resource burden of model creation. This problem is outside of the scope of the first question, which presupposes that the models are built from data for a particular police force. In this second question, the models are built with PF1 data then used on PF2 data.

\subsubsection{Are models accurate over time?} Language changes both through the introduction of new words and as a result of changes in the usage of existing lexical units. In policing contexts, officers can also be encouraged to record different facts over time. These changes could potentially change the form or the wording of an MO text. If the language of MO texts changes so as to differ from the language that the PTMs are finetuned on, then one can expect performance to deteriorate. Although only two years’ worth of data are used here, this hypothesis is tested by finetuning a model on data from one period and testing it on data from a subsequent period. Understanding how or when the performance of a model may deteriorate is important for ensuring that the model that is being used has been trained correctly.

\section{Data} The data that are used in this study are from PF2. They were described comprehensively in Chapter 8. The PF2 data were whitelisted by the project team in order to remove personally identifying information. This process was also described in Chapter 8. The main difference from the PF1 data results from the addition of victim characteristics, namely ethnicity and sex, as metadata. This addition is conducive to a more profound investigation of the potential biases that the PTMs produce.

Beyond victim characteristics, additional details were provided after the models had been built for validation purposes. Links to stolen vehicles were added in order to facilitate the validation of the vehicular theft model. A link is an entry in the police database that indicates whether a vehicle was stolen during a given burglary. This provides an additional verification that enables the performance of the finetuned PTM to be assessed. The completion of the database link results in structured data that is easy to search. PF2 analysts expect “stolen” links to have higher completion rates than flags (the structured data that were introduced earlier). A “stolen” link is therefore an appropriate structured indication of whether a vehicle was stolen. It can be compared to the model classification of the text data.

The PF2 data also contained more details about the date and time of the offences that had been committed. The PF2 data included details on the years, months, days, and times of offences, whereas the PF1 data included only months. Consequently, the data on dates can be analysed in greater detail. As an interesting aside, the PF2 data also cover the period of the initial Covid-19 pandemic in the UK, including the first lockdown. The effects of that lockdown on intra-crime variation for burglary can also be observed.


\section{Methods} The methods of this study were introduced in Chapter 9 and recapped in Chapter 10, which is on Study 1a. The general process is similar to that which was explained previously. The deviations from the approach that was described in the general introduction are listed below.

\subsection{Labelling} As in Study 1a, the fine-tuning of the PTMs is a supervised learning process. Therefore, labelled data are required for the models. The data were labelled by two researchers, with the author holding the casting vote in the event of disagreement. The MO text was selected through an active learning strategy, as detailed in Chapter 9. On this occasion, the labelling data pool was limited to burglaries committed between October 2018 and the end of 2019 (as mentioned in the data chapter, October 2018 coincides with the introduction of the new data-recording system for PF2). This restriction was introduced in order to facilitate the investigation of the accuracy of the model over time, that is, to enable the third research question of the study to be answered. Active learning was only conducted for the motor-vehicle model. Accordingly, all PTMs were fine-tuned on data that had been selected for the motor-vehicle model through active learning. In total, 1,982 MO texts were read and labelled for the burglary classification models.

\subsubsection{Fine-tuning Models} Finetuning PTMs entails training the model with a focus on data that are specific to the classification problem at hand, that is, on the MO data. There were no significant differences between the finetuning methods that were applied to the PTMs in this study. Finetuning was completed by using the same methods as those that were outlined in Chapter 9. The BERT-large model was used consistently. The hyperparameters were all set in the manner that is described in Chapter 9.

\subsubsection{Performance} The additional data fields that are provided with the PF2 data allowed for a deeper investigation into bias than had been possible with the PF1 data. Bias against individuals of certain sexes and ethnicities was explored by comparing PTMs across different victim characteristics. Bias was explored by using the metrics Equality of Opportunity (EoO) and Predictive Parity (PP), both of which were introduced in the methods chapter. EoO is based on recall and measures the disparity of the probability of a TP across groups. For example, given that a classification is positive, what is the probability of finding it? PP is based on precision and is a measure of the disparity of the probability of FPs across groups. For example, given that the model finds that a classification is positive, what is the likelihood that the classification is correct?

These metrics were calculated for each test set, and a cross-validation experiment was completed. As noted earlier, a reference group was selected for each bias in order to determine whether there is a difference between the reference group and the remainder of the population. The reference groups were \say{white European} and \say{male}, and they were compared to the groups \say{all other ethnicities} and \say{females}, respectively. Unknown and missing values were excluded from the analysis.  \footnote{The analysis was also conducted with these missing values included in the comparison groups and there was no significant difference in the result.}. 

No comparison to a basic keyword model was conducted. The advantages of the PTMs over the keyword approach were explained in Chapter 5 and demonstrated in Study 1a. However, it was possible to make a comparison with another method that police forces may use. Police forces often record some aspects of intra-crime variation as flags. Flags are typically key words or phrases that a police officer can select to describe a crime. In the PF2 data, these flags had been selected from a series of dropdown menus on the crime-recording software. These flags are much easier to search than free-text data because they are structured and are therefore be used often to find crimes of interest. The following process was followed in order to compare the flags to the model: firstly, a decision was made about the flags that describe classification types accurately. For example, it was determined which flags highlight burglaries in which force was used. The list of flags that were used for each classification is displayed in Table \ref{tab:burg_keywords}. Secondly, the monthly counts of crimes that do and do not meet the criteria of the classification were summed. This step was completed for both the crimes that were selected by reference to flags and to the crimes that were selected through the use of the NLP model. It was possible to compute monthly percentages of positive classifications from the sums of the positive and the negative classifications. Finally, the percentage of positive classifications was plotted as a time-series line, and the line plots were compared.


\begin{table}[]
\centering
\begin{tabular}{p{0.3\linewidth}|p{0.6\linewidth}}
\toprule
Classification                    & Flags                                                                                     \\ \midrule
\multirow{4}{*}{Motor vehicle}    & Instrument Used, Key Used, Stolen                                                            \\
                                  & Instrument Used, Key Used, Key Used                                                          \\
                                  & Property, Conveyance, Car                                                                    \\
                                  & Property, Conveyance, Motorcycle                                                             \\ \midrule
\multirow{2}{*}{Force}       & Trademarks- Attack Method Premises                                                           \\
                                  & Entry Method, Attack Method Premises                                                         \\ \midrule
\multirow{2}{*}{Outbuilding} & Location, Garage - Includes premises for sale and repair but does not include petrol station \\
                                  & Domestic|location, Garden - Driveway, Shed                                                   \\ \bottomrule
\end{tabular}
\caption[Burglary Keywords]{\label{tab:burg_keywords}The keywords used to filter the MO Keywords data column in the PF2 Burglary data. }
\end{table}


As mentioned in the data section, there was an additional validity check on the motor-vehicle classification, namely for the presence of a link between a stolen vehicle and the crime. These data were also used to check the validity of the vehicular theft model and were added to the monthly time-series plots that were described in the previous paragraph. The calculation method was the same.   

\subsubsection{Model Performance Over Time.} Model performance over time was investigated by fine-tuning the model on data from one period and testing it on data from a subsequent period. All PTMs were fine-tuned and initially tested on data from the period between late 2018 and the end of 2019. For the replication element of this study, all MCC metrics were gathered from a test set that was randomly selected from the same set of dates. However, a separate test set was also built for 2020, which allowed the model from the earlier time period to be tested on the 2020 data. It is possible that the 2020 data are not ideal for comparative purposes due to the Covid-19 pandemic. The pandemic resulted in severe mobility restrictions as the government tried to curtail the spread of the virus, and it expanded the lexicon of the general public. However the effect of the pandemic on burglary MO texts may have been less severe because it is not immediately clear how the pandemic would change burglary methods and, therefore, the words that the police use to describe burglaries. However, if there is a significant degradation in model performance over the years, then the Covid-19-induced variation would be one source of change that would require further investigation.   

\subsubsection{PTM transfer-ability} PTMs that were fine-tuned on PF1 data were used to label the PF2 test sets. MCC scores were calculated and directly compared to the MCC scores from the models that were built from the PF2 data. For example, the force model that was built from the PF1 data was used to label the PF2 test set, and the labels that were generated were compared to the force labels. Comparing the two MCC scores indicates how accurate a model that is generated by one police force can be when it is employed by another police force.

\section{Results} The results of the replication study are presented first. The MCC scores and explainability are directly comparable to the earlier study of the PF1 data because the two studies are based on the exact same methods. The estimates of bias are different because the PF2 data cover victim characteristics. Therefore, the metrics of extrinsic bias for the sex and ethnicity groupings were examined. The comparison with the NLP labels for which the police recorded keywords are displayed as line plots after the bias results. Thereafter, the exposition turns to the MCC results for the change in time period and the reuse of models across forces.

\subsection{MCC} The MCC results are presented in Table \ref{tab:lancs_mcc}. The table refers to the test sets from 2018–2019 and from 2020–2021. However, before these sets are explored, it is necessary to explain, in brief, how much data were labelled and why.
 
 As detailed in the methods chapter, when the active learning strategy was used, labelling for the validation set would cease when MCC exceeded 0.9. The motor-vehicle model, which was selected for labelling first,  never achieved this value (see Table  \ref{tab:results_1c}). For the reasons that are given in the next paragraph, additional labelling was unlikely to result in increases in MCC. Consequently, all labelling ceased after the 16th active learning batch. 

The motor-vehicle task was selected for labelling first. However, all tasks (i.e., the force used and outbuilding only tasks) were labelled at the same time. At Batch 16, enough texts had been labelled to gauge the necessity of additional labelling for the two other classification tasks. The force used classification task had achieved an MCC of 0.92 by Batch 5 (see Table \ref{tab:results_1c}). Therefore, no further labelling was necessary.

The outbuilding task did not reach an MCC of 0.9 by Batch 16; the MCC scores stabilised at approximately 0.85 at Batch 5 (see Figure  \ref{fig:mcc_burg_lancs}.). It was therefore unlikely that additional labelling would increase the MCC score. However, a single batch of additional active learning was conducted, with a fine-tuned outbuilding model applied to the final selection. The MCC of the model (tuned on 16 batches of motor-vehicle and one batch of selected outbuilding data) did not increase as a result. Therefore, the author determined that no further labelling was necessary. The data from this 17th batch are omitted for simplicity.

\begin{table}[]
\centering
\begin{tabular}{@{}lcccccccc@{}}
\toprule
Batch       & 1    & 2    & 3    & 4    & 5    & 6    & 7    & 8    \\ \midrule
Motor vehicle          & 0    & 0.48 & 0.65 & 0.73 & 0.56 & 0.82 & 0.8  & 0.82 \\
Force       & 0.52 & 0.71 & 0.82 & 0.88 & 0.92 & 0.93 & 0.88 & 0.92 \\
Outbuilding & 0.33 & 0.23 & 0.72 & 0.84 & 0.87 & 0.85 & 0.85 & 0.86 \\\midrule
Batch       & 9    & 10   & 11   & 12   & 13   & 14   & 15   & 16   \\\midrule
Motor vehicle         & 0.75 & 0.72 & 0.88 & 0.86 & 0.82 & 0.75 & 0.82 & 0.72 \\
Force       & 0.88 & 0.92 & 0.92 & 0.9  & 0.92 & 0.91 & 0.93 & 0.91 \\
Outbuilding & 0.85 & 0.86 & 0.85 & 0.84 & 0.86 & 0.84 & 0.85 & 0.86 \\ \bottomrule
\end{tabular}
\caption[Batch metrics - PF2 data. All models]{\label{tab:results_1c}MCC values (based on the validation set) for models fine-tuned on PF2 Burglary data. Batch refers to the active learning batch e.g. after 5 batches of labelling (500 MO texts) the motor vehicle model had an MCC of 0.56 }
\end{table}


The MCC scores for the models are reported in the subsections that follow. An MCC score of 1 is optimal, while a score of 0 is equivalent to a finding of randomness. The results that are discussed below are confined to the 2018-2019 test set. The 2020-2021 test set scores are discussed at a later stage in the exposition. 

\subsubsection{Motor Vehicle model} As explained the motor vehicle model was selected as the first model for labelling via the active learning strategy. The values of the MCC after each active learning batch, calculated using the validation data, can be seen in Figure \ref{fig:mcc_burg_lancs} and in Table \ref{tab:results_1c}. The highest value attained was 0.88, which was below the requirement for the stop condition (0.9). The labelling was stopped after the sixteenth batch after it had become apparent that there were no more positive classifications within the pool of potential training data. In other words the active learning strategy had already selected all MO texts with a motor vehicle theft for the training data, there were no more positive samples to learn from in the potential training data. In fact the last positive example had been found in batch 11. From the plot at Figure \ref{fig:mcc_burg_lancs} it is clear that the additional negative examples, forming batch 12 to 16, did not aid the model fine tuning and so further fine-tuning was not deemed necessary. Consequently model fine-tuning stopped at 16 batches and the model was tested on the test set.

\begin{figure}[!tbp]
  \centering
    \includegraphics[width=\textwidth]{images/mcc_burg_lancs.png}
    \caption[MCC scores for the PF2 burglary models.]{{MCC scores for the PF2 burglary models.} MCC scores are shown after each iteration of the active learning strategy. The Force and Outbuilding models peak relatively early on at batch 5 and 6. Whereas the motor vehicle model peaks at 11. Some of the variation will be attributable to the random initialisation of the models. Source: Author generated.}
    \label{fig:mcc_burg_lancs}
\end{figure}


MCC scores on the test set can be found at Table \ref{tab:lancs_mcc}. The fine-tuned model over the 10 runs had a mean MCC score of 0.98, near perfect performance. On half of the runs the model correctly classified each of the 200 MO texts from the test set. The MCC metrics for the motor vehicle model are comparable to the scores from the motor vehicle model built and used on the PF1 data (mean of 0.97). 

\subsubsection{Force model} The force model used the same data labelled from the active learning conducted on the motor vehicle model. The mean MCC score from the final ten initialisations was 0.93 on the 2018/19 test set. These scores are higher than the MCC score for the validation set, indicating that perhaps the validation set had difficult to classify MO texts.  These MCC results are comparable to those models fine-tuned and tested on the PF1 data (mean of 0.91). 


\subsubsection{Outbuilding Model} The MCC scores for the outbuilding model on the test data were again better than the validation set scores. The mean of the ten initialisations on all of the 2018/19 test set is 0.90. This is the lowest score across all three models. The outbuilding model was not built with the PF1 data as the data was not suitable. Therefore there is no direct replication for the outbuilding results presented in this study.



\begin{table}[]
\centering
\begin{tabular}{@{}lcccccc@{}}
\toprule
         & \multicolumn{2}{c}{Motorvehicle} & \multicolumn{2}{c}{Force} & \multicolumn{2}{c}{Outbuilding} \\ \midrule
Run      & 18/19           & 20/21          & 18/19       & 20/21       & 18/19          & 20/21          \\
1        & 1.00            & 0.89           & 0.93        & 0.92        & 0.91           & 0.94           \\
2        & 1.00            & 0.93           & 0.94        & 0.93        & 0.90           & 0.93           \\
3        & 0.94            & 0.88           & 0.93        & 0.94        & 0.90           & 0.94           \\
4        & 0.97            & 0.93           & 0.93        & 0.95        & 0.89           & 0.93           \\
5        & 0.94            & 0.88           & 0.94        & 0.96        & 0.92           & 0.94           \\
6        & 1.00            & 0.91           & 0.90        & 0.93        & 0.91           & 0.94           \\
7        & 0.97            & 0.88           & 0.90        & 0.94        & 0.85           & 0.96           \\
8        & 1.00            & 0.91           & 0.92        & 0.95        & 0.90           & 0.92           \\
9        & 0.97            & 0.90           & 0.92        & 0.96        & 0.91           & 0.94           \\
10       & 1.00            & 0.90           & 0.95        & 0.96        & 0.89           & 0.92           \\\midrule
Mean     & 0.98            & 0.90           & 0.93        & 0.94        & 0.90           & 0.94           \\\midrule
Best Run & 1.00            & 0.93           & 0.94        & 0.96        & 0.92           & 0.96           \\ \bottomrule
\end{tabular}
\caption[Final model MCC metrics. PF2 data. All models.]{\label{tab:lancs_mcc}MCC values (based on the test sets) for models fine-tuned on PF2 Burglary data. Scores are generated from 10 separate fine-tunes based on all labelled data. 18/19 refers to the test set from only the years 2018 and 2019, similarly 20/21 refers to the years 2020 and 2021.}
\end{table}




\subsection{Explainability} LIME was again used to understand how the words of the texts are contributing to the final classification. As a reminder BERT uses the word and the surrounding context of the word, so a global understanding of how the model works is difficult to ascertain. LIME provides a local understanding for each MO text by randomly deleting words to see what effect they have on the final classification. This is scaled up in this thesis by utilising LIME on all MOs in the test set then using word clouds to show the most important words based on the individual LIME model coefficients. Word clouds for the motor vehicle, force and outbuildings model can be seen in Figures \ref{fig:wordcloud_mv_both_lancs} ,\ref{fig:wordcloud_force_both_lancs} and \ref{fig:wordcloud_home_both_lancs} respectively.

\begin{figure}
     \centering
     \begin{subfigure}[b]{0.9\textwidth}
         \centering
         \includegraphics[width=\textwidth]{images/mv_wordcloud_positive.png}
         \caption{Words that contributed to a positive classification}
         \label{fig: wordcloud_mv_lancs}
     \end{subfigure}
     \vfill
     \begin{subfigure}[b]{0.9\textwidth}
         \centering
         \includegraphics[width=\textwidth]{images/mv_wordcloud_negative.png}
         \caption{Words that contributed to a negative classification}
         \label{fig: wordcloud_mv_rev_lancs}
     \end{subfigure}
        \caption[Wordclouds from  \textbf{motor-vehicle} classification model. PF2 data.]{{Wordclouds from  \textbf{motor-vehicle} classification model. PF2 data.} These wordclouds were generated using a fine-tuned BERT model on the PF2 data. The larger a word the more important it is for a classification. Words size is derived from a summation of the coefficients from individual LIME models. Word sizes are not comparable across figures. Source: Author generated.}
        \label{fig:wordcloud_mv_both_lancs}
        
\end{figure}


\begin{figure}
     \centering
     \begin{subfigure}[b]{0.9\textwidth}
         \centering
         \includegraphics[width=\textwidth]{images/force_wordcloud_positive.png}
         \caption{Words that contributed to a positive classification}
         \label{fig: wordcloud_force_lancs}
     \end{subfigure}
     \vfill
     \begin{subfigure}[b]{0.9\textwidth}
         \centering
         \includegraphics[width=\textwidth]{images/force_wordcloud_negative.png}
         \caption{Words that contributed to a negative classification}
         \label{fig: wordcloud_force_rev_lancs}
     \end{subfigure}
        \caption[Wordclouds from  \textbf{force} classification model. PF2 data.]{{Wordclouds from  \textbf{force} classification model. PF2 data.} These wordclouds were generated using a fine-tuned BERT model on the PF2 data. The larger a word the more important it is for a classification. Words size is derived from a summation of the coefficients from individual LIME models. Word sizes are not comparable across figures. Source: Author generated.}
        \label{fig:wordcloud_force_both_lancs}
        
\end{figure}




\begin{figure}
     \centering
     \begin{subfigure}[b]{0.9\textwidth}
         \centering
         \includegraphics[width=\textwidth]{images/home_wordcloud_negative.png}
         \caption{Words that contributed to a positive classification}
         \label{fig: wordcloud_home_lancs}
     \end{subfigure}
     \vfill
     \begin{subfigure}[b]{0.9\textwidth}
         \centering
         \includegraphics[width=\textwidth]{images/home_wordcloud_positive.png}
         \caption{Words that contributed to a negative classification}
         \label{fig: wordcloud_home_rev_lancs}
     \end{subfigure}
        \caption[Wordclouds from  \textbf{outbuilding} classification model. PF2 data.]{{Wordclouds from  \textbf{outbuilding} classification model. PF2 data.} These wordclouds were generated using a fine-tuned BERT model on the PF2 data. The larger a word the more important it is for a classification. Words size is derived from a summation of the coefficients from individual LIME models. Word sizes are not comparable across figures. Source: Author generated.}
        \label{fig:wordcloud_home_both_lancs}
        
\end{figure}




The word clouds for the motor vehicle model demonstrate a similar pattern as those from the PF1 data, see Figure \ref{fig:wordcloud_mv_both_lancs}. Firstly the most prominent words in the word cloud for a positive classification (i.e. a motor vehicle was stolen) are words that a human might expect to use when completing the same classification task. The three most important words being \say{car}, \say{vehicle} and \say{keys}. Also note that these words are disproportionately important, they are much larger than the remainder of the words. This contrast with those words contributing to a negative classification. Figure \ref{fig:wordcloud_mv_both_lancs} panel B, has word sizes that are much more even across all words selected. The words for the negative classification have no observable theme, this is likely because it is not written in the MO when a car is not stolen.

Word clouds for the force model are similar for the positive case (force was used). The model is using words that are commensurate with what a human might use for the same classification task. Although some of the more important verbs are less prominent, and there appears to be more of a focus on nouns than the PF1 data. Overall though the pattern of important words is clear and logical. Unlike the motor vehicle model, the negative classification, no force used, is often reported and so there is a clear pattern to the second word cloud. Figure \ref{fig:wordcloud_force_both_lancs} panel B. Here the word insecure is the most important word for obvious reasons. \say{Unknown} is also prominent as it is often used to say that the method of entry was unknown, reflecting a lack of obvious force used to enter the building. 

The outbuilding word cloud is similar in structure to the motor vehicle model. The positive classification cloud has a smaller amount of disproportionately important words, here \say{shed}, \say{Garage} and \say{garden} being the most important words. The negative classification word cloud has words that are more equitable in size, and in general the word cloud represents words that are encountered throughout all Burglary MOs. Again this is likely because the  negative classification is not explicitly reported.

Each of the word cloud pairs has given confidence that the model is using the words that a human might use in determining the classification of each text. This gives confidence that the models are picking up on the correct features of the text, and not spurious correlations. The next section investigates any bias that the models may have in relation to sex and ethnicity.

\subsection{Bias} Bias within the models was investigated along sex and ethnicity characteristics. The reference groups were male and white-european respectively. Models were investigated by exploring extrinsic bias metrics, equality of outcome and parity of prediction. Table \ref{tab:lancs_bias} has the results from both the models built on the active learning data and also the 10 fold cross-validation experiment models. The results are described in relation to the partition of the data i.e sex and ethnicity rather than by model. As a reminder 0 is no bias, a positive number is bias in favour of the reference group and a negative number is bias against the reference group. The maximum and minimum possible values are 1 and -1 respectively.

\subsubsection{Ethnicity} There are two significant p values from the cross-validation experimentation and they are both for ethnicity. One is for equality of outcome for the motor vehicle model, the other one is for predictive parity for the force model. In both cases the mean shows that the bias is very slightly against the reference group, that is white Europeans are potentially discriminated against by the models. 

When reviewing the results of the model built from the active learning data we see that most values of both equality of outcome and predictive parity are close to zero indicating little bias. Four out of six of the metrics are negative for the equality of outcome and five out of the six predictive parity metrics are negative, again both indicate a slight discrimination against white Europeans. The results are mixed across the models. Bias is only consistent in one direction for the outbuilding model. However, even then bias for outbuilding model does not register a statistically significant result with the cross-validation experiment. In summary there are small values of bias, but not consistent results to indicate systemic bias for PTMs classifying MO texts in the PF2 data.
 
\subsubsection{Sex} The evidence for sex bias is weaker still, there are no statistically significant results from the cross-validation experiments. The predictive parity has an equal number of negative values as it does positive values. For equality of outcome there is one more value of positive than negative. In conclusion there is no evidence of sex bias from the PTM when classifying MO texts for the PF2 data.


\begin{table}[]
\begin{tabular}{@{}llcccc@{}}
\toprule
\multicolumn{6}{c}{\textbf{Equality Of Outcome}}                                                                                                                  \\ \midrule
Model         & Partition & \multicolumn{1}{l}{Actual 18/19} & \multicolumn{1}{l}{Actual 20/21} & \multicolumn{1}{l}{CV Mean} & \multicolumn{1}{l}{CV p value} \\\midrule
Motor vehicle & Ethnicity & 0.000                            & -0.040                           & -0.049                   & 0.001*                      \\
Motor vehicle & Gender    & 0.000                            & -0.129                           & 0.002                    & 0.902                       \\
Force         & Ethnicity & -0.022                           & 0.056                            & 0.004                    & 0.593                       \\
Force         & Gender    & 0.048                            & 0.014                            & 0.004                    & 0.530                       \\
Outbuilding   & Ethnicity & -0.012                           & -0.017                           & -0.005                   & 0.462                       \\
Outbuilding   & Gender    & -0.017                           & 0.001                            & -0.004                   & 0.147                       \\\midrule
\multicolumn{6}{c}{\textbf{Predictive Parity} }                                                                                                                   \\
Model         & Partition & \multicolumn{1}{l}{Actual 18/19} & \multicolumn{1}{l}{Actual 20/21} & \multicolumn{1}{l}{CV Mean} & \multicolumn{1}{l}{CV p value} \\\midrule
Motor vehicle & Ethnicity & 0.000                            & -0.077                           & 0.155                    & 0.078                       \\
Motor vehicle & Gender    & 0.000                            & -0.005                           & 0.045                    & 0.060                       \\
Force         & Ethnicity & -0.040                           & -0.040                           & -0.108                   & 0.024*                      \\
Force         & Gender    & 0.091                            & -0.120                           & 0.001                    & 0.948                       \\
Outbuilding   & Ethnicity & -0.012                           & -0.017                           & -0.003                   & 0.821                       \\
Outbuilding   & Gender    & -0.032                           & 0.009                            & 0.000                    & 0.989                       \\ \bottomrule
\end{tabular}
\caption[Bias Metrics. PF2 data. All models.]{\label{tab:lancs_bias} Extrinsic bias metrics for the Lancashire Burglary models. AL refers to the model built with data selected by active learning, the following digits represent the year of the test set. The mean refers to the mean result from the 10 cross-fold validation experiment. The p value relates to the hypothesis test that the mean, from the cross-fold experiment, is not zero. * is for a p value that is significant.}
\end{table}

\subsection{Flag Comparison} This section compares the NLP generated labels for the three models with flags that the police may search in order to identify the intra-crime variation of interest. In addition the presence of links between stolen vehicles and the burglary is explored i.e. burglaries that have a stolen vehicle linked to that crime.

\subsubsection{Motor Vehicle Model} The time series plot for the motor vehicle model can be seen at Figure \ref{mv_ts}. The police generated labels - both \say{Linked vehicle} and \say{Flagged} should only be fully considered after 2019 because of the aforementioned change in data recording systems. The two striking elements of the plot are firstly that the NLP labels and the linked vehicle labels are very well matched (Pearson correlation coefficient of 0.94) and secondly that the flags return much fewer crimes. This low return from the flags is repeated throughout the classifications. In discussions with the analysts from PF2 they recognised that the flags do not have a high completion rate, though based on their experience they thought that the linked vehicles data would be completed to a high standard. 

To explore the differences between the NLP model and the linked stolen vehicles an error analysis was conducted. The error analysis reviewed one hundred of the MOs where the NLP model had identified a motor vehicle stolen but there was no linked vehicle. In total there were 432 errors of this kind. Of the one hundred MOs checked - 63 did have vehicle stolen in the MO texts and so the NLP model was correct.The remainder (37) were labelled incorrectly by the PTM. The majority of these errors was where only the vehicle keys were stolen and not an actual vehicle (21). Although these latter errors maybe useful in the context of this crime, because keys can be used later to steal a vehicle,  they are not what the model was trained on.   


\begin{figure}
  \includegraphics[width=\linewidth]{images/mv_linked_time_series_plot.png}
  \caption[Motor vehicle model time series plot]{A time series plot of the motor vehicle classification. Showing data generated form the PTM model (NLP), linked vehicles and flags. Source: Author generated.}
  \label{fig:mv_ts}
\end{figure}

\subsubsection{Force Model}The force model only shows a comparison between the flags and the PTM labels. The plot is at Figure \ref{fig:force_ts}. As with the previous time series plot the most notable finding is that the PTM  finds many more burglaries with the use of force than the flag system produced by police officers. Again this finding is consistent with the analysts view that the flag system is not well used. The next notable finding is the apparent seasonality in the PTM labels. Here it can be seen that the proportion of crimes where force is used to enter the building is consistently higher in the winter months than the summer months.   


\begin{figure}
  \includegraphics[width=\linewidth]{images/force_time_series_plot.png}
  \caption[Force used model time series plot]{A time series plot of the force used classification. Showing data generated form the PTM model (NLP) and flags. Source: Author generated.}
  \label{fig:force_ts}
\end{figure}


\subsubsection{Outbuilding Model} The outbuilding model plot only shows the PTM generated labels alongside the police generated flags. The plot is at Figure \ref{fig:outbuild_ts}. As with the other two plots the PTM returns more crimes with the positive classification, although the number returned here are closer than the other two plots. The spike in the two time series in early 2020 coincides with the first Covid-19 lockdown in the UK. This spike may indicate a proportion shift in burglary type as a result of lockdown policies.


\begin{figure}
  \includegraphics[width=\linewidth]{images/outbuilding_time_series_plot.png}
  \caption[Outbuilding only model time series plot]{A time series plot of the outbuilding only classification. Showing data generated form the PTM model (NLP) and flags. Source: Author generated.}
  \label{fig:outbuild_ts}
\end{figure}

\subsection{Model transfer-ability} This section of the results reports on the usage of models from one police force area in another police force area. The results are for  the use of PF1 models on PF2 data, the reverse was not possible due to data security limitations. The MCC results in Table \ref{tab:results_transfer} show that the models are reasonably transferable. In each case the MCC is lower for the transferred model, which as one would expect, but the drop is not that significant in all cases. This demonstrates that models built in one area will have some utility in another force area. The implications of which are discussed later.  


\begin{table}[]
\begin{tabular}{@{}llcc@{}}
\toprule
\multicolumn{1}{c}{Test Set} & \multicolumn{1}{c}{Model} & PF2 Model PF1 Data & PF1 Model PF1 Data \\ \midrule
18/19                        & Motor vehicle             & 0.93                   & 0.98                   \\
20/21                        & Motor vehicle             & 0.80                   & 0.90                   \\
18/19                        & Force                     & 0.91  & 0.93  \\
20/21                        & Force                     & 0.90  & 0.94 \\ \bottomrule
\end{tabular}
\caption[Model metrics. Models tested on alternate police force.]{\label{tab:results_transfer} MCC scores for the use of models built with PF1 data and used to classify PF2 data. PF2 metrics included for comparison. }
\end{table}

\subsection{Performance over time} Test sets were built for 2018/19 and 2020/21, even though the training data only came from 2018/19. This allowed a view of how model performance would vary over time. The results in Table \ref{tab:lancs_mcc} show the results of the 10 model initialisation using the active learning data to fine-tune the PTM. The mean result of the ten initialisations is reported here. For the motor-vehicle model there is a sizeable drop in performance from a 0.98 to 0.90, although a performance of 0.9 may still be adequate depending on usage.  For the force model there is an increase, but it is very small from 0.93 to 0.94 (also note the the 20/21 test set has a higher top scoring run than the 18/19 set). The outbuilding model also increase this time by a larger amount going from 0.90 to 0.94.


\section{Discussion} This section synthesises the results presented in the section above in relation to the main research questions given at the start of the chapter. Each question is explored in turn with comparisons to the original study with the PF1 data.

\subsection{Can the results in Study 1a be replicated in a different police force?}Study 1a set out to understand if PTMs can be utilised to classify MO texts. The two classification tasks were 1)Was a motor vehicle stolen during the burglary? 2) Was force used to enter the building during the burglary?  Additionally Study 1a explored whether the PTMs would be explainable and therefore generate trust that the models were working as human might do, and not relying on perhaps spurious correlations in the data. In study 1a bias was studied to a limited extent due to a lack of victim characteristic data.

Performance results in the replication study were equivalent to the results in the original study, both produce high MCC scores indicating high performing models are possible from fine-tuning PTMs. In addition in the replication study an additional classification problem was explored, that of burglaries into out-buildings only. A model was also fine-tuned on this problem and the resulting model also showed good performance with a high MCC score.

Comparing the labels generated from the PTMS to police generated data we find that the PTMs return much more crimes. Combined with the high MCC scores, and the error analysis with the linked data, this gives confidence that this return is a more accurate reflection of the intra-crime variation. Again this highlights the benefits of the PTMs over existing police processes for exploring intra-crime variation. 

To test explain-ability the LIME model was also used to generate word clouds which showed the most important words for each classification model. As with the PF1 models, the replication study produced word clouds that enhance trustworthiness. The important words highlighted by these clouds were entirely consistent with the words that a human may use to make the classification judgement and therefore they again give confidence that the model is classifying in a way commensurate with how a human would classify the texts.

In the replication study there was more victim characteristic data than the original study, so the models could be explored for bias along ethnicity and sex characteristics. The results showed that there was no evidence of systematic bias in the classifications of the fine-tuned PTMs. It should be noted that within the texts there was very little reference to the particular characteristics mentioned i.e. victim sex and ethnicity generally was not described or referenced to. This means that any bias was likely to be introduced indirectly through systematic variation in language and or quality of the MO rather than explicit mentions. From the bias investigation in the original study, 1a,  we note that length of text and percentage of BERT words was not correlated with either of the bias metrics and so, even if for instance Asian victims had short MO texts the model would not necessarily perform poorly against those MOs. 

As a reminder there \emph{may} be a number of different routes to introduce biases in the chain that leads from a crime being committed to the formulation of a MO text and the subsequent classification. Firstly the crime may not be recorded as the victim may prefer not to interact with the police therefore there is no MO text. If the victim does interact with the police the interaction might be sub-optimal (e.g. due to language barriers) and so the information passed might not accurately describe the crime in full. Finally the PTM is built on data that has been scraped from the internet, this data is almost certainly likely to reflect biases in everyday society and so may perpetuate these into the classifications. The bias investigation in this study can only pass comment on the final route - the use of the PTM. The first two routes are beyond the scope of this study as already explained. The bias results found here therefore indicate that any biases that are inherent in the PTM are not affecting the classification of burglary texts for the problems investigated. 

In summary the replication study has replicated the good results from the first study and has extended them proving that PTMs offer good performance in the classification of burglary texts. Additionally the models are classifying the texts using similar words that humans would use offering evidence that the models are working in a trustworthy manner. From the limited investigations of bias conducted, there is no evidence of systematic bias in the model classifications.


\subsection{Can models trained in one police force area be used in another force?} By replicating the first study in a second police force area the opportunity arose to use the models trained in one police force on the data of another police force. If models performance is carried from one force to another then model utility becomes more enhanced as models can be re-used across forces without the need to share data. The results have shown that models can be transferred from one police force to another and retain a reasonable level of performance. The implications are that models can therefore be shared across forces for direct use or to seed the start of fine-tuning of a separate model and therefore reduce the labelling burden. This may have important practical implications if for instance the knowledge for classifying a model is relatively specialised, for example the detection of modern day slavery. 

That models can be reused across forces also has implications for PTM implementation. In the UK for instance there are 43 police forces, all with similar crime recording techniques, a common language and resource pressures. If models can be shared then a central repository of models would therefore have utility across all forces. A central repository of models would allow maximum sharing and a commensurate reduction of the labelling burden. Additionally the technical aspects of model running and fine-tuning could also be housed centrally reducing the training burden across the 43 forces. Extension of model sharing to this extent would need much more experimentation than is offered here, with a sample size of just two. Never the less the results presented here are encouraging.


\subsection{Are fine-tuned PTMs accurate over time?} As language use changes so will eventually the performance of the models. The language in a MO text is reflective of the intra-variation of a crime, so as the variations changes, for instance in response to a new security technique, then so will the language in the MO texts change. Models will therefore have to be checked to ensure that they remain relevant for the language used. In this study the models were trained on data from one year the tested on data from a subsequent year. There was no perceptible drop in performance in either of the three classifications tasks. This gives confidence that the models are robust to some time change, and this during a change of significant social upheaval i.e. the Covid-19 pandemic. 

However, there is no evidence to suggest that, despite a general decline in burglary, here was any new type of intra-crime variation. Variation that may have changed the language being used in the second time period used for this experiment. This means that whilst we have limited evidence that models are robust to the passage of time, there is no evidence about the robustness to new criminal techniques and therefore a change in language. Changes like this will have to be guarded against and the findings here certainly do not exclude the necessity to check the validity of fine-tuned PTMs over time.


\section{Conclusions}This replication study has provided additional evidence that PTM are able to effectively classify Police MO texts by extending the problem to another police force area and one more additional classification task. In addition the study was able to extend the investigation into bias by showing that there was no evidence of bias in classification on either a gender or an ethnicity basis. The replication study also investigated the utility of models over time, finding no perceptible drop in classification power. This indicates the models will remain useful over extended periods. However, the study was relatively weak,  and a more thorough study would be required for a definitive assessment of the refresh rate for models. 

In addition, models were  shown to be effective on the same problems but on data from a different police force. This suggests model sharing between forces may be possible. Model sharing would considerably lower the labelling, computational and skills burden required to use PTMs. This may prove to be an important aspect with respect to the practicality of implementing PTMs because it would significantly reduce resource costs. It could also indicate that centralised coordination, and perhaps development, of some aspects of work, would be an efficient approach. 

These studies have shown that PTM can be effective with MO text data, across a number of different classification problems. However MO texts are not the only text that police forces have. Another style of text are police incident logs, that encompass crime and non-crime data. The next case study will progress this work  by using PTMs to classify anti-social behaviour incident logs.
 






