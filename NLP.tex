\chapter{Natural Language Processing}

Natural Language Processing (NLP) is a branch of artificial intelligence that seeks to extract information from text, principally free text. The main purpose of NLP is to \say{make human language accessible to computers} \parencite{eisenstein2018natural}. This is generally accomplished by accepting data as words and then representing them in the form of numbers. Once the data is in the form of numbers it can then be manipulated by the ML processes outlined in Chapter 4.


Within NLP there is a tension between knowledge (trying to understand the structure of language) and learning (using algorithms to efficiently code representations with a focus only on the results) \parencite{eisenstein2018natural}. The knowledge advocates have been working on language problems for some time under the guise of computational linguistics. They have laid down many important NLP foundations that can be used to automatically extract information from text, such as part of speech tagging and parsing a sentence so that the dependencies between words can be understood. As \textcite{manning2015computational} notes, there has been a shift in NLP more recently to those with more of a focus on the learning. That is, they want to use modern machine learning processes, and especially deep learning processes, to translate raw text directly into the desired output \parencite{eisenstein2018natural}.

This research aims to leverage this shift in research focus and utilise the most effective NLP models based on the learning construct. In particular to utilise a class of models known as PTMs. PTMs, introduced earlier, are powerful NLP models  because they already have an element of language understanding built-in before they are used on a specific task. 

The analogy used in the introduction was PTMs were like employing a graduate and conducting job specific  training, rather than earlier NLP models which required full training in an area before they could be used. Most of these PTMs have been pre-trained on edited text, such as news reports or other structurally published material. For that reason, the models have been built on text that is likely to have a different compositional structure than police free text data, and so PTM utility with this data is not obvious.

Figure \ref{fig:overview} gives a generic language processing pipeline that might be used to gain information from free text. The focus of the previous chapter was on the machine learning models to the right of the diagram. This chapter, however, comes before machine learning in the process and is concerned with the processing of the data – that is, finding an appropriate representation of the text in the form of numbers.

A few notes on terminology. A token is an individual element of interest, which generally will be a word, but it can be a piece of punctuation. It is the lowest level of investigation. Tokens are collected to produce a document. A document in this research is a single description of a MO for one crime; however, other examples are a single tweet or, in certain cases, a whole book. A collection of documents is called a corpus. In this research, the corpus will be a collection of MO data, all from the same police force and of the same crime type.


\begin{figure}
  \includegraphics[width=\linewidth]{transfer_figs/Slide9.jpeg}
  \caption[NLP process Overview.]{This figure demonstrates a generic machine learning task containing NLP. The exact details of the tasks will be explored in this chapter. Source: Author generated}
  \label{fig:overview}
\end{figure}


This section begins with a note on different applications for NLP, then moves onto techniques that can be used to harmonise and understand the individual tokens in a document. The section then progresses to how these tokens can be represented within a document and how the differences between documents within the same corpus can be mapped. Finally pre-trained language models (PTMs) will be introduced. PTMs are the most advanced kind of NLP model and are the model that this research will be based upon.

\section{Applications}

Natural language processing has many applications, just like machine learning in general. However some important applications are as follows:

\begin{itemize}

\item{Classification.} Classifying documents into one of several categories can be general, such as a positive movie review, or more specific, such as a MO where the offender has used a knife.

\item{Information Extraction.} This may be to extract the disease from clinical notes, without knowing exactly what disease it is you are searching for.

\item{Question and Answering.} In this application questions are asked of a specific corpus and the answer is returned. In this application both the question and the corpus may need to be subjected to NLP techniques to generate the answer.

\item{Translation.} Translating form one language to another.

\item{Chatbots.} Where computers are designed to respond to conversations with humans.

\end{itemize}

For this research the focus will be centred around classification, as it is generally considered a gateway task before moving onto more complex applications.


\begin{figure}
  \includegraphics[width=\linewidth]{transfer_figs/Slide12.jpeg}
  \caption[Bag-of-Words Example.]{This figure demonstrates how a bag-of-words algorithm operates. The tokens in each document are counted and a matrix is formed with a column for each word in the corpus and a row for each document in the corpus. If document 1 contains the word in col 2, then a 1 is placed in the cell (2,1). Two techniques to provide a mores succinct output are also shown. Stemming, which reduces word forms, and stop word removal which removes common words of little value. Source: Author generated}
  \label{fig:BOW}
\end{figure}

\section{Text Normalisation} On of the most simplest forms of NLP output is what is known as the bag-of-words modelling. This method produces, an unordered, representation of all the words in a given document by producing a matrix with \emph{1} if the word is present and \emph{0} if the word is not present, Figure \ref{fig:BOW} gives a toy example. This matrix, called a word-document matrix, can then be used as input to machine learning algorithms. Some elements to note here are that the order of words is not kept, so some of the semantics of the language can be lost. Secondly, with slight different variation in word forms, even a small selection of basic sentences can give rise to a relatively large matrix, this makes it harder for the computer to grasp the meaning of the words, is \emph{cat} really that much different from \emph{cats} that they need separate columns? Reducing the size of the matrix will also make the computation more efficient as there is less matrix manipulation to conduct. What follows is a brief exploration of the techniques to reduce the variation in tokens within a document to help convey the same or very similar meaning with less tokens.

\subsection{Stemming} Stemming is the process of removing inflectional affixes from a word. Examples of inflectional affixes include the plural marker \emph{s} and the past tense marker \emph{ed} \parencite{eisenstein2018natural}. See Figure \ref{fig:BOW} for an example. Stemming groups words with the same underlying concept and so reduces the total number of different tokens in a document and corpus, allowing similarities between tokens to be more easily identified. As an example horse, horses and horsed all become hors once stemmed. Note the stem in this instance is not a real word, but this would not affect the algorithm \parencite{jivani2011comparative}. Stemming is conducted through a rules based system, and so on some occasions, the stemming generated is not correct.  An error can either be over-stemmed, where two words of differing meaning are given the same stem – for example, Williams to William – or it can be under-stemmed, where two words with the same meaning have different stems – for example, tooth and teeth.

\subsection{Lemmatizing} Lemmatization is an additional step that can be used to reduce the amount of individual tokens in a word-document matrix. It is similar to stemming but attempts to avoid some of the pitfalls of the rule-based system by additionally understanding the context of the word. One of the more popular lemmatizers, WordNet \parencite{MillerGeorgeA1990ItWA}, is a lexical reference system, similar to a database or dictionary, which lists words and there synonyms under a joint lemma, this means that WordNet can be interrogated with a given word and its part of speech, such as noun or verb, and after a look-up, a lemma will be returned. The system can make less mistakes than stemming, as the rules are hardcoded; however, computationally it can be more expensive.

\subsection{Stop words} Stop words are words that are so common (e.g. \emph{the}, \emph{a} and \emph{to}) that they are generally thought to play little role in the linguistic meaning of a document. Stop words are corpus dependant, so while there are lists of common stop words it is best practice to tailor each list to the text at hand. As an example we can see that even after the removal of some classic stop words in Figure \ref{fig:BOW}, \emph{have} appears in all documents. If this was a real example then there would be serious consideration for including \emph{have} as a stop word as it does not assist in the discrimination between documents. 

Normalising text can have its advantages. The meaning of a document can be distilled to a smaller size, and additional rules and dictionaries can be leveraged to clear some ambiguities. However, removal or changing of a token can reduce the information that is left in a document, information which may be useful for further discrimination of the NLP model. Bag-of-words models in general lose all their semantic value as the word order is lost, and so these techniques are particularly useful for those situations where semantic importance is not high.

\section{Word Features}

 \subsection{Part of Speech Tagging}Another method for enriching the data set is to understand what part of the syntax of a document each token represents. That is, a part of speech (POS) tagger will label each word as a noun, verb, etc. By labelling the words in this way, some ambiguous meaning can be avoided. Take the following headline as an example: \say{Dealers will hear car talk at noon}. If \emph{talk} in this example is a noun, then there are no surprises; however, if it is a verb, then we may question what kind of dealers they are and what they have been doing with their produce. A popular and effective  \parencite{zeman-etal-2018-conll} open-source tagger - UDPipe \parencite{straka-2018-udpipe} – can label an English sentence with the correct POS tags with around 90\% accuracy. These models have been built using labelled data from the universal dependencies treebank. In this tree bank 37,000 English sentences have been hand-labelled with their POS tags. This labelled data has then been used to generate the open-source model UDPipe. Part of speech tagging is useful for understanding text in general; however, it can have more specific uses, such as finding entities like names or addresses, in a process known as named entity recognition (NER). This is described next.
  
\subsection{Named Entity Recognition} As the name suggests, NER helps to draw out information from text relating to real world entities, be they people, places or organisations  \parencite{eisenstein2018natural}.More recently, the types of subject entities have been widened to include drugs, medical conditions and different types of biomedical items such as protein types  \parencite{goyal2018recent}. NER is an important step in many NLP applications because it helps to draw out salient information between document types. These named entities, once discovered, can form part of the feature engineering of a document and be linked across documents as additional information. A popular open NER model is the Stanford NER \parencite{finkel2005incorporating}. This model was trained on newspaper reports that have been manually tagged. The data a model was trained on will have implications for its use outside of that domain. As \textcite{prokofyev2014effective} show models trained outside of highly specialised domains show significant drops in their effectiveness, as such testing of open-source models and possibly adaption is required on new styles of corpus.


\begin{figure}
  \includegraphics[width=\linewidth]{transfer_figs/Slide13.jpeg}
  \caption[Sentence Dependancy Parse.]{This figure demonstrates the dependancy parsing of two similar sentences. Both parses produce a similar structure of dependancies despite the difference in the wording. \emph{suspect} and \emph{hammer} both join to \emph{smash(ed)} before they reach the \emph{window} in both parses. Source: Author generated}
  \label{fig:dep}
\end{figure}


\subsection{Sentence Parsing} An additional measure that can be taken with sentences within documents is to parse them so that the internal dependencies are understood. Dependency parsing takes a sentence then produces a dependency for that sentence, beginning at the root of the sentence and cascading to all words within it. The root is decided by a set of deterministic rules dependant on the type of sentence and the word types within it  \parencite{eisenstein2018natural}.  Two examples of a dependancy tree, the result of sentence parsing, are shown in Figure  \ref{fig:dep}. Knowing the dependencies between words is useful, both for information extraction but also question and answering tasks, because understanding the underlying dependencies in a sentence can help clear some of the ambiguities that were introduced from the manner in which it was presented. The clarity sentence parsing can bring is seen in Figure \ref{fig:dep} where the two sentences, with essentially the same meaning, have very similar dependancies despite the difference in wording. We see that both the \emph{suspect} and the \emph{hammer} have to pass through \emph{smash(ed)} before they reach the \emph{window} in both cases. An additional important application of sentence parsing is negation, whereby it is important to tract where a negative clause is acting. 


\section{Word Representation Methods}


\epigraph{\centering You shall know a word by the company it keeps}{J. R. Firth}


In this section so far there has been an introduction to how to normalise the text by reducing the amount of individual tokens in a document, then building on that by understanding what features can be extracted by dependency parsing, NER and POS tagging. Explored here is the meaning of the individual words, how the meaning contributes to the totality of the document’s meaning and how similar words can have similar meanings across documents.


\begin{table}[]
\centering
\begin{tabular}{@{}lcccc@{}}
\toprule
\textbf{Token} & \multicolumn{1}{l}{\textbf{TF}} & \multicolumn{1}{l}{\textbf{DF}} & \multicolumn{1}{l}{\textbf{IDF}} & \multicolumn{1}{l}{\textbf{TF-IDF}} \\ \midrule
pet  & 1 & 10  & 1   & 1    \\
dog  & 2 & 50  & 0.3 & 6.65 \\
cat  & 2 & 100 & 0   & 0    \\
Fido & 1 & 1   & 2   & 0.5  \\ \bottomrule
\end{tabular}
\caption{\label{tab:search} Example TF-IDF values for a 100 document corpus.}
\end{table}

\subsection{Frequency Methods}

The bag-of-words model utilises the simplest representation of words, the words presence or absence is noted by a binary marker (generally 1 or 0) see Figure \ref{fig:BOW}. This method does not draw any explicit meaning from the word itself, it only marks its presence. The next stage up from this is to change the binary marker to a count maker, so that the number of times the token is present in a document is now recorded, giving additional weight to multiple uses, although as there is a tendency of words to cluster this method tends not to show much improvement on the simple binary choice \parencite{eisenstein2018natural}, this is known as term frequency. 

An additional method utilising the same approach seeks to understand how important a word is in that document, given its prevalence in the corpus as a whole. This method is known as TF-IDF, which stands for term frequency – inverse document frequency. The first part, term frequency, was outlined above and is a count of the terms in that document. The second part, the inverse document frequency, is a measure of the number of documents that the word is mentioned in. It is inverted because words that are rare in the corpus should have more discriminatory power  \parencite{manning2008introduction}. Typically a logarithmic measure is used. 

Each of the procedures outline above was demonstrated with single tokens; however these procedures can be generalised to groups of tokens which represent phrases. Groups of tokens are known as \emph{n-gramms} where  \emph{n} relates to the number of individual tokens in a phrase. An example of a tri-gram is \emph{New York City}. n-grams can either be used exclusively or alongside single tokens so that common phrases can be extracted for greater fidelity, especially useful, if as with the example above, the n-gram is a single entity. 

The examples above have distilled the information from each token into a document down to a single number, and the presence or size of that number contributes to the meaning of that document within that corpus, along with the distribution of the other tokens. That single number represents that word. What this process still not allow for, however, is the individual meanings of words, as opposed to their mere presence, to contribute to the characterisation of the document. For this reason, word embeddings were developed that would more accurately contribute to the meaning of individual words. These are discussed next. 


\subsection{Word Embeddings}

The general idea behind word embeddings is to represent each word with a vector of numbers (typically, they can go as high as 300 numbers for one word) such that words with similar meanings have similar vectors. Either these embeddings can be generated for individual corpuses, or previously derived embeddings can be used. Typically, these derived embeddings have been trained on a massive corpus, such as the whole of Wikipedia.

\begin{figure}
  \includegraphics[width=\linewidth]{transfer_figs/Slide14.jpeg}
  \caption[Word Embeddings visual example]{Left panel shows vector offsets for three word pairs illustrating the gender relation. Right panel shows a different projection, and the singular/plural relation for two words. In high-dimensional space, multiple relations can be embedded for a single word. Source: \textcite{mikolov2013linguistic}}
  \label{fig:word}
\end{figure}

Embeddings are generated in a number of different ways, but in essence they exploit the same relationship given in the quote at the start of this section – that is, a word is defined by those around it. A single word of interest may occur in a corpus a number of times, but if it is conveying the same meaning, the words surrounding it will be similar. Embeddings are created by investigating the probability of seeing a neighbouring word, given the target word  \parencite{mikolov2013efficient}. Those target words with similar property distributions for the same neighbouring words, will have similar meaning. The property distributions are encoded into the vectors of interest and relationships and similarities can be easily found. Figure \ref{fig:word} demonstrates how these vectors and their relationships can be explored visually. The main downside of these vectors though is that they can not differentiate between homographs (words that are spelled the same but have different meanings e.g. river \emph{bank}  and money \emph{bank}). Recently other models have been introduced that are able to reduce this problem by modelling the context of the word. The most promising of these models are BERT \parencite{devlin2018bert} and GPT-2 \parencite{radford2019language}. These models are known as PTMs.


\section{Pre-Trained Language Models} PTMs are a particular class of language models. They are also referred to as Large Language Models (LLMs) or foundational models. As mentioned in the introduction to the thesis and the introduction to this chapter, PTMs are different to the normal classes of models that have been introduced so far. The main difference with a PTM is that it has already been partially trained to understand language. That is PTMs are firstly \emph{trained} to understand a language before they are \emph{fine-tuned} on a specific task, for example classifying burglary MOs.

A useful analogy for understanding PTMs is a university student embarking on their first graduate job (PTM) compared to someone without education (generic ML model). The training to be successful at the job will have two parts. First, the trainee receives broad formal education,  culminating in a university degree. They have a lot of knowledge and understand broad concepts, but it has taken many years and lots of effort to get them to that point. When they reach their new job, they will need additional, job-specific training tailored to the problems they need to solve for that role. The graduate needs additional domain knowledge, which will build on the broad concepts that they already understand. However, this additional knowledge is quicker to impart because of their already broad understanding of the underlying concepts. In the case of humans the cost of the education is somewhat dependant on the amount of people to be educated, however humans have a significant drawback that computers models do not have - they can not be easily replicated. As PTMs can easily be replicated (you can download a copy of one from the internet in minutes) that upfront training cost is only bourne once.

Using a PTM is like employing a graduate for the first time. The model already has some understanding, or knowledge, of the problem. In this case, the problem is what English words mean. However, the model does not understand the specific problem well. Therefore, the model must be given some on the job training before it is released for work. Previously, one could not \say{employ a graduate}, one had to do all the model training oneself. Now, with the introduction of PTMs, one can \say{employ a graduate} and so skip most of the training.  This means that significant complexity and effort in using NLP models has been removed from the end user. The first part of the training for PTMs is called pre-training and the second part is called fine-tuning. In this work only the fine-tuning will be conducted.

In addition to this two part training the PTMs also have a mechanism, called attention, that allows the model to understand context. Broadly attention allows the PTM to understand how the context of a word effects the meaning of another word in the sentence or the overall meaning. For instance the presence of \say{river} next to {bank} will lead the model to representing the sentence as a waterway rather than a financial institution. Likewise the presence of \say{not} near \say{good} is likely to lead to a more negative sentiment than a positive one.

PTMs are deep models, they are based on layers of neural networks that are trained to modify the input to output the correct result. As mentioned in Chapter 4, deep models are useful because they can reduce the need for feature engineering. Feature engineering highlights the most important features of a model input. The hard part of feature engineering is knowing which features to engineer and then finding a suitable representation. Examples of feature engineering were given above e.g. PoS tagging and NER.   Feature engineering is time consuming and so by using deeper models the effort to establish a model is reduced. The downside to deep models however is that explaining why a model has made a decision is more difficult. Expainability techniques are therefore required to understand how and why deep learning models including PTMs are making decisions.

As PTMs offer the best performance across a range of NLP tasks they will be the model type used in this research. The exact PTMs and how they are built will be detailed in the methods chapter. 

\section{ NLP Conclusions} This section has demonstrated that there are modern techniques available to assist with the extraction of information from unstructured free text data. These techniques have, for the most part, been developed into open-source models (PTMs) that can produce state of the art results on the material for which they were trained. These open models are coupled together into a processing pipeline, which relies on the success of each step to produce a numerical representation of the subject text that can then be explored through the use of the aforementioned ML techniques.

However, when these models are used outside of the types of data they were trained on, their efficiency drops. This is especially true if the underlying structure or grammar, of the text changes. As the focus of this research will be centred on police data, the next chapter will survey which and to what extent NLP techniques in this section have utilised with police generated free text data and what kind of benefits they produced. It will be shown that NLP techniques have been used to good effect on police data - although the practice is not widespread. The next chapter also highlights a gap in the research, in that PTMs have not been used with police free text data, and so their performance in this area is unknown.

