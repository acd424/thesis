

\chapter{Problem Oriented Policing}

As described previously POP is a policing model, introduced in 1979, to ameliorate some of the short comings stemming from the traditional policing model of response. The general aim is to tackle the factors that allow a crime opportunity to occur, so that it no longer has the opportunity to occur in the future. In this way the overall aim is crime (or problem) prevention.  The next section will explain in more detail the core principles of POP. This will be followed by a thorough exposition of SARA an analytical framework for the conduct of POP to demonstrate how it is conducted in practice. Finally there will be an evaluation of the utility of POP before assessing POPs weaknesses.

\section{Overview of POP}

Problem-oriented policing was introduced by Herman Goldstein in 1979 \parencite{gold79} as a new policing model to replace the traditional response policing style. Since its introduction POP has been widely utilised as a method to reduce problems faced by police services across the world \parencite{fairnessandeffectivenessinpolicing_2004, stockholmlec}. The central thrust of POP is to focus on the ends of police activity (e.g. reductions in harm to the public) rather than the means (e.g. number of convictions). Recognising that the police have a wide variety of objectives to deal with, the focus should be on the resolution of these problems, not on the means or the ways of addressing these problems. 

\begin{figure}
  \includegraphics[width=\linewidth]{transfer_figs/Slide5.jpeg}
  \caption[A schematic of POP.]{A schematic for POP. Reproduced from \textcite{eck1987problem}}
  \label{fig:POP}
\end{figure}

The mechanism for this prevention is demonstrated in the schematic at Figure \ref{fig:POP}. The incidents are generated from an underlying causes, however in the traditional response model the incidents are responded to individually and typically, due to resource constraints not all incidents are either known about or can be resolved. The POP model takes the information from these responses and the similar incidents and aims to tackle the underlying causes so that the opportunity to create the problem no longer exists \parencite{eck1987problem}. In broad strokes this is how POP aims to reduce harms, by using knowledge of similar incidents then altering the crime triangle so that the conditions are no longer present for the problem to occur.

Police business is not just about crime, it is about all problems that the police are responsible for, or are thought to be responsible for. In order to help with this there is a specific definition of what a problem can be \say{Problems are a cluster of harmful incidents that the public expects the police to handle} \parencite{popchap11}. These problems though should have a common theme, so that they can be grouped and addressed together. Although most of the focus may be on crimes, POP does not exclusively focus on crime and recognises that the police remit is much wider than crime alone.  


The next few sections will introduce and explain some of the core principles surrounding POP. When considering these principles its important to keep in mind that \say{POP is a framework, or methodology, for addressing police problems and not an intervention strategy per se} \parencite{scott2020problem}, that is to say that not all of these principles are directly required for the solving of \say{\emph{a}} problem, but they are required to build a culture of problem solving in general. 


\subsection{Focus on Harm Reduction}POP places more emphasis on preventative responses rather than remedial, enforcing the age old heuristic that \say{prevention is better than cure}. Remedial action, acts of immediate response, investigations and arrests, is the staple diet for the traditional model of policing, POP seeks to move away from these as default and to act before the crime or problem arises. That’s not to say that these elements do not have their place, the emphasis is to rebalance the focus between remedial and preventative action \parencite{goldstein1990}. 

This focus on harm prevention switches the measure of effectiveness of the police, it moves away from more managerial approaches of clear-up rates and arrests statistics to a more considered view of the polices effectiveness around reducing problems\parencite{goldstein1990}.  \textcite{eck1987problem} suggest that there are a number of ways that effectiveness in POP can be measured, this does not rest solely on problem elimination, but also in the reduction of similar incidents, the seriousness of those incidents or in how the incident is responded to. Note that the focus is squarely on the incidents and the characteristics of the future occurrences, frequency, severity etc. The focus is not on the more traditional metrics of police success such as arrests or response times.

The focus on POP is not catching criminals  but in preventing problems. If problem opportunities are not presented then incidents can not occur in the first place, and criminals can not cause harm to the population to be protected, and as we can see from situational crime theory this is a real possibility that must be explored. However as we have seen from Figure \ref{fig:POP} grouping of incidents with the same underlying cause is necessary for the efficiency of reducing many incidents from a single intervention. This is the focus fo the next principle. 

 
 
 \subsection[Specificity! Specificity! My Kingdom for some specificity! ]{Specificity! Specificity! My Kingdom for some specificity! \footnote{With apologies to Shakespeare’s \emph{Richard III}.} }


As we have seen above, the central idea behind POP is to reduce incidents of harm by disrupting the underlying causes of the problem. This disruption is achieved through grouping of individual incidents in to problems, understanding the similar mechanisms that cause the problems then disrupting these mechanisms so that the problem can no longer occur. 

Identifying these groups or clusters of problems can be difficult and as \textcite{scott2012implementing} highlights there is a tension between breadth and depth of knowledge of problems in organisations. In short the higher one goes up the organisational hierarchy the wider one can see problem occurrences, they gain greater breadth of the situation and therefore the efficiency of POP can increase as more single incidents can be grouped. However this increased breadth is at the expense of the detailed knowledge of each problem which can be found lower in the hierarchical order allowing each group of problems to enjoy greater intra-similarity. This tension between breadth and depth of problem knowledge can inhibit the optimal implementation of preventative measures \parencite{maguire2015problem}.

Once the incidents have been grouped it is then necessary to understand the separate incidents. This examination is not necessarily about individual elements, but more about the similarities of mechanisms between the individual acts that make up the problem set. The focus is more on the 'why' than the 'what' or 'who' of traditional policing. Why did this problem arise? How did the circumstances around each problem set the conditions for the harmful act to occur? What are the common factors between problems? Answering these questions with fine-grained analysis leads to a deeper understanding of the problem itself. Understanding the steps and conditions that lead to the problem mean that points can be identified and tackled to prevent the conditions for the harmful act being realised, reflecting routine activity theory and the principles behind situational crime prevention \parencite{felson1998opportunity}.

As \textcite{felson1998opportunity} highlight, \say{crime opportunities are highly specific}, that is they should be understood and grouped by how they have been committed and not necessarily what the outcome of the problem was. The keystone element for POP therefore is specificity. Specificity in fitting a solution to a well developed problem. Specificity is both the Achilles heal and the Herculean strength (apologies to readers for the mixing of ancient metaphors) of POP. So although POP is incredibly effective, it is also difficult to achieve. Where effort in specifying the problem falls short, this directly influences the effectiveness of the solution and hence the final result \parencite{maguire2015problem}. This is why any effort to make the analysis of a problem easier, more effective or more efficient will have a disproportionate effect in the success of POP.


\subsection{Tailored Responses} 
 
The microscopic evaluation of the problem allows a new approach to be taken for each problem. Each problem will undoubtedly have its own set of conditions and unique factors, and by understanding these a new and problem specific strategy can be developed to tackle that particular problem. This is the main thrust of the approach. Pick a solution that is effective for the problem set at hand, which is achieved through understanding the problem thoroughly.

 Set against a back drop of the traditional policing model of responding to crime incidents, POP sought to expand the repertoire of police responses by encouraging the use of tools other than the criminal justice system. Criminal justice systems can be slow and inefficient, and may not do a good job of ameliorating the harm that has occurred.  With a focus on prevention it is necessary to look outside the traditional toolbox of police responses to find a new set of tools. This new set of tools will allow the leverage of other capabilities in the public and private sectors that can be utilised to change the conditions that allow problems to flourish. Reflecting on POP in 2018, \textcite{stockholmlec}, reflects on the success and \say{enormous potential} of the use of non-police entities to reduce crime by using their powers or resources. 
 
Formulating tailored responses is resource intensive as the problems need to be extensively detailed and a fitting solution found. In order to make the formulation of the response less onerous there is a heavy emphasis on reporting and logging results so that inspiration, though not exact solutions, can be used to formulate tailored responses. 
 
 
 
\subsection{Evaluate the results}

With a rigorous focus on reducing harm, it is important that POP has within its framework an emphasis on evaluating how well its achieving its stated aim. There needs to be an understanding of which POP implementations have worked, and which have not and why. This not only helps to ensure that the actions are having the desired consequences, but also it helps to make the implementation of the process more efficient, by building a body of knowledge that can be used by all practitioners. 
 
Proving that something has not happened, a counter-factual, is always more difficult than demonstrating an occurrence. Additionally attributing that non-occurrence to a specific intervention can be even harder. That is why the evidence for the utility (positive or negative) of POP must be actively sought. The measurement must begin at the outset and may even need to encompass an area much wider than the target zone. Measurement will be difficult to achieve in retrospect alone. Identifying weak signal in noisy environments is difficult, so the use of analytical techniques that are not routinely found in the police organisations is likely to be necessary thus adding to the analytical burden \parencite{popchap11}.  

In addition to understanding internally whether a POP intervention has been effective it is also necessary to publicise the results. As we have seen the introduction of POP is a change to the norm, it is not the de facto style and it is not what most officers in police forces envisioned they would be doing when they joined the organisation. Reporting the results is crucial to building an understanding of if POP works and therefore is a worthwhile activity for the police to engage in.

The results reported will help to build a body of knowledge about what works. Given the focus on specificity of problems, solutions are unlikely to be able to be ported wholesale from one area to the next. However building a body of knowledge is important for two reasons. Firstly it will allow some of the analytical burden for each round of problem solving to be completed a little quicker, drawing on the experiences of others will allow adaptations of plans, or a swifter understanding of mechanisms that can then be adapted. Secondly the body of knowledge will act as a beacon for the effectiveness of POP and be a fulcrum for the turning of the tide of institutional resistance.
 
 
\subsection{POP Summary} Returning to Figure \ref{fig:POP} we recap what the essence of POP is. POP is about changing the underlying conditions that allow crimes or problems to flourish. These changes are brought about through applying analytical power to first group problems, analyse their structure to find a suitably specific response. Once this response has been implemented there is further need to document the effect and report the results to contribute to a wider body of knowledge to contribute to the understanding and efficient conduct of POP.  This section was about what POP is, the next section will take a deeper look at how POP is conducted.

\begin{figure}
  \includegraphics[width=\linewidth]{transfer_figs/Slide6.jpeg}
  \caption[The SARA problem-solving process.]{The SARA problem-solving process. Source \textcite{clarke2003becoming}}
  \label{fig:SARA}
\end{figure}

\section{SARA  - An Analytical Framework for POP} Although POP can be implemented by police forces in a number of different ways \textcite{scott2012implementing} suggests there are two broad implementations of POP in a police force. Either to have all Officers conduct POP, the generalist approach, or to build specific capability and units  using a more specialist approach. No matter how POP is implemented the broad analytical process followed tends to be centred on what is known as the SARA process \parencite{POPUCL}. SARA stands for Scan-Analyse-Respond-Assess and the cycle is shown in Figure \ref{fig:SARA}.  Clearly depending on the type of implementation for POP depends on the depth to which the SARA process can be used, however there is a general flexibility within the model to account for those differences. The POP guide ``Become a problem-solving crime analyst in 55 steps"  \parencite{clarke2003becoming} is a key document for the implementation of POP in the UK, and sets out how the SARA process should be followed. What follows is a brief look at the four stages of SARA, as described in \say{55 Steps} and how they interact to form the lifecycle of a problem solving process.


\subsection{Scan For Problems} The first stage in the process is to scan for a problem, and here it is worth remembering exactly what a problem is. In his book \textcite{goldstein1990}  Goldstein defines a problem as:

\begin{enumerate}

\item A cluster of similar, related, or recurring incidents rather than a single incident.
\item A substantive community concern.
\item A unit of police business.

\end{enumerate}


This definition is quite broad, but it does allow for an understanding to be formed about what one should be looking for when they are searching for problems. Particularly that the problem must be reoccurring and have a negative affect on the community. Dependant on the type of POP implementation in an organisation(generalist or specialist) depends on what scanning horizons will be used. Clearly if POP is disaggregated throughout the force (generalist) then many sensors will be picking up on smaller collections of problems but to a finer detail. Where POP is more centralised (specialist) the view of the POP scan will be much wider, but will suffer from a lack of detail, because either the information required is recorded but is hard to access or it is simply in the heads of the Officers closest to the problem \parencite{goldstein1990}. Such trade-offs are inevitable in large organisations, but being able to either widen a scanning horizon, or detailing more information about each problem is likely to move closer to lessening the severity of the trade-off required.

The scan in \say{55 steps} is focussed heavily on defining the problems once they have been found in the scan phase - but this presumes that the problems have already be identified. If the problem is a unit of police business then the sub-element has to be a single incident (as seen in Figure \ref{fig:POP}). The scanning phase identifies these incidents then characterises the incidents in order to group them into a single problem. Once incidents have been characterised and grouped into similar problems then the job of more clearly defining the boundaries of the problem can begin in the next stage. It is important to note at this stage that although problems are focussed on harms to the community \textcite{maguire2015problem} highlights that most of the routes through which cases are nominated for POP action are through police data (70\%), meaning that extracting and stratifying police data is likely to lead to improvements to problem identification and formulation.

\subsection{Analyse in Depth} The problem has been selected and the boundaries of what is considered the problem have been broadly defined, now it is time to fully understand the problem to refine development.  It is at this stage that the detail in order to understand the specifics of the problem are developed, it sets the problem apart from others and lays bare the underpinning processes and factors that generate the opportunity for the problem to exist. 

This stage requires all aspects of the problem and its incidents to be understood\parencite{clarke2003becoming}. This will broadly ensure trying to understand all of the actors involved in the problem, this will include the more obvious examples of victims and offenders, but also other actors that might be identified from the problem triangle, such as offender handlers and place managers. In addition to understanding the actors, knowing the contexts of the incidents including any important physical or social factors that led up to or resulted in the problem will help to identify similarities and pinch points where preventions can be directed. 

In order to acquire the information to characterise the problem, POP practitioners should consider a variety of information sources, this should include. The established literature, the POP centre \footnote{https://popcenter.asu.edu} has a wide range of literature that includes specific problem guides as well as academic articles on POP successes. Police files, which includes the full gambit of documents including witness statements and forensic reports, will be a vital source of information, though they do have their drawbacks as they very often do not reflect the whole problem process. Additionally to these written sources, speaking with the original police officers that dealt with the incidents, the victims, witnesses and the offenders can be a rich seam of information to enable problem understanding \parencite{goldstein1990}. 

Understanding problems to this level of detail requires a concerted analytical effort, that can not easily be found in a Police Force that is not geared towards an analytical approach. This analytical burden is reflected in the results of \parencite{POPUCL}, where results showed that analysis in POP investigations frequently only included one type of analysis - rather than the three highlighted above, and most investigations barley moved above a cursory exposition of simple crime count data.  To further highlight the problem over half of the respondents said they lacked enough analysts to complete this phase properly \parencite{POPUCL}. 

\subsection{Respond} Once the Problems have been found and deciphered so that a thorough understanding of what is happening and how that is contributing to creating the conditions for crime to occur


Find a practical response is how it is framed in \say{55 Steps}  and they draw heavily on the five methods of situational crime prevention - mentioned  above - to begin to systematically investigate responses.  Of note here is the POP centre webpage \footnote{ \url{https://popcenter.asu.edu/all-problems} } which includes seventy-four different problem solving guides highlighting those responses that have been used before and found to work. The responses need to be appropriate and need to be aligned with the work found in the analysis stage. POP puts an emphasis on non-enforcement activity and preventative measures. These potential measures are not limited to those actions that the police can conduct themselves, but are part of a wider community approach.  


\subsection{Assess} As set-out earlier assessing the effectiveness of responses employed is beneficial for both proving that POP works, but also for detecting success, which may otherwise not be as obvious as traditional methods. It is also important to look at any diffusion of benefits that may have occurred around the target area, or within the target area but of a different nature. Again these metrics can be hard to grip and comparing them to other areas may be necessary to highlight differences to the counter-factual situation where the intervention did not occur. Noisy data may make it especially difficult to pick out weak signals, and changes may not fall across existing recording criteria.  All these factors mean that thorough problem and technical knowledge will be required before, during and after an intervention to ensure that the full impact of the intervention is known. 

The SARA framework is a cycle and as practitioners come to this point in the cycle they should begin again from the beginning - building on their knowledge of the problem further refining the detail and then any additional required responses will make best use of the analytical framework.

%\section{Other models of policing} POP has been well defined and explained in some detail, but there are other methods of policing that whilst not necessarily in direct competition with POP, do compete for resources within police forces for implementation.  \parencite{tilley2008modern} highlights three main modern models and they are briefly discussed below to highlight the main differences to POP.
%
%\subsection{Community Policing} This model of policing does not have an explicit harm reduction mandate, the model is more concerned with  \say{facilitating a two-way communication between the police and the public} \textcite{tilley2008modern}. The exact definition, as Tilley notes, is fairly hard to pin down, it is certainly not as well defined or advocated for as POP is. Community policing is seen as something that increases or reinforces the legitimacy of the Police, at least locally, and is a method to ensure that those being policed have a say in how it operates.  A systematic review into community policing, \parencite{gill2014community}, shows no strong effects for crime prevention, though they note that that is not its primary purpose. They also do not see a statistically significant increase in effectiveness when(albeit limited) problem solving is utilised alongside community policing.
%
%\subsection{Intelligence-led policing} This model of policing is focussed on say{doing the practical business of policing more smartly, incorporating modern information technology and modern methods.} \parencite{tilley2008modern} .That is it is centred around law enforcement and the more traditional role of policing, though it priorities gathering of intelligence for preemptive enforcement strikes, rather than a purely responsive attitude. Intelligence led policing is more developed institutionally than POP, in the UK at least as here the National Intelligence Model has been rolled out to all police forces and explicitly endorsed by the Home Office. It is seem more as a business process model, that makes better use of information and information flows to realise a more efficient system, however the end goals are still centred around enforcement. That is not to say however that more efficient information gathering processes and flows cannot benefit POP, indeed some, notably  \textcite{kirby2004integrating},have suggested that the NIM will provided a much firmer base from which to launch POP efforts.


\section{Success of POP} POP is generally regarded as a successful method. There have been two Campbell systematic reviews into the effectiveness of POP and both have shown that overall POP is successful in reducing the problems it has set out to tackle. 

The first review, \parencite{popeffective}, showed a moderate indication of success. Although only the strict meta analysis was conducted against ten studies they found a small but positive effect in favour of POP. Most comparisons however were against the standard model of policing, though they did account for additional resources where appropriate. The research papers did not compare directly against other policing models such as Intelligence led or community policing.  

Additionally because the reviewers found so few studies that met the criteria for the systematic Campbell review they also reviewed 'before and after' studies that didn't meet the full criteria.  In reviewing the additional forty-five 'before and after' studies they found reductions in problems of up to around 35\%, that is if the standard model was allowing 100 problem incidents to happen in a unit time then after a POP intervention this may reduce to somewhere between 70 and 80 incidents. Of course what this doesn't account for are the net benefit of less problems, and this would be hard to quantify, but if less incidents are being responded to then more police resource is available for other tasks ( perhaps even more crime prevention). Also if hypothesis such as the debut crime hypothesis and the keystone crime hypothesis  \parencite{farrell2015debuts} are true then the benefits over the longer term for some problems will almost certainly be larger as the effects compound. 

The updated review,  \parencite{hinkle2020problem}, was able to include many more studies in the main analysis (34) as the quality of the formal evaluation process has increased in the ten year interval between the two studies. This second review has found an even larger effect, using the stricter criteria,  and even managed to quantity the benefits in diffusion with no crime displacement from POP activities. That is that areas surrounding the POP interventions generally also saw a net decrease in those problems. 

Another review this time into hot-spot policing, \parencite{braga2014effects}, also found a further reduction in crime when problem solving techniques were used alongside hot-spot techniques. As a control they used studies that used hot-spot policing coupled with a more traditional approach. 

We have demonstrated here that POP can be successful, and indeed generally is. But what is represented by these studies amounts to an analysis on a \say{per-protocol basis} that is only those that followed the protocol (SARA) were measured in the study, in practice a more thorough picture of the merits of POP would be based on an \say{intention-to-treat} basis that would highlight where POP could of been used and why it hadn't either been used or been effective.



\section{Impediments to POP} 



Although POP has been shown to be effective in reducing crime, the fact that only thirty-four effective surveys in forty years of policing were available for the second Campbell review can, and probably should, be seen as evidence for a lack of widespread and sustained adoption. In fact many studies and reviews of POP have found that although POP reduces crime, it is difficult to implement. In an accompanying article to the first Campbell review mentioned above \textcite{whitherpop} cites three reasons for POP not working as well as people may have initially hoped. These three areas are explored below.



\subsection{Weakness 1 - The conduct of POP}  



The conduct of POP largely relates to adherence to the SARA procedure and in particular the issue of analysis and specificity when dealing with the problem at hand. In \textcite{scott2012implementing} , the need to both get and train the right staff (Chapter 9) but also for enhanced analytical support (Chapter 17) is highlighted at great length. The conduct of POP requires appropriate  knowledge, skills and experience to be delivered effectively, but because these skills are not required for the traditional response policing model, there is therefore currently a lack of these skills in police forces. 

To chronologically bookend this point a lack of analytical skills was identified by Goldstein as early as 1990, \parencite{goldstein1990}, and was still seen as an issue in 2016 \parencite{popchap11}. The review of POP in England and Wales \parencite{POPUCL} concluded that \say{recurrent weaknesses in the application of SARA...concerned the depth and quality of problem analysis.}, additionally they also found that \say{43\% of survey respondents said they did not have access to information necessary to perform effective problem-solving}. That the crux of POP lies in the understanding of the problem at hand, yet the police forces that want to implement POP do not have those skills sets available in sufficient quantities, it is hardly surprising that the conduct of POP can be sub-standard.  However it is encouraging to note that it would largely appear to be a resourcing issue, rather than a systemic POP problem as where analytical resourcing has been sufficient, largely as a result of collaborations with academia, POP success have been strong. 

If some analysis could be automated, or partially automated,  then at least one bottle neck to further implementation would be widened. As will be shown later, modern NLP techniques combined with machine learning , has the potential to allow the rapid exploitation of police free text information, if this information can be proved to have utility in the POP process then it is likely to contribute to lowering the analytical burden for a successful POP implementation. 


\subsection{Weakness 2 - The Delivery of POP}


\epigraph{\centering Baldrick:  But this is a sort of a war, isn't it, sir?
 
Blackadder: Yes, that's right.  You see, there was a tiny flaw in the plan.
 
George:   What was that, sir?
 
Blackadder:  It was bollocks.}{Blackadder Goes Forth: Goodbyee }


With all the best intentions of a plan and an analytical strategy - if they are not thought through formulated and tested against the practicality of delivery then they are bound to fail even on seemingly mundane issues. As we see with Blackadder explaining to Baldrick about the precarious peace treaties built before the First World war, they looked good, they sounded good, but had anyone really stress tested the plan to ensure it would work. Were all parties committed to the plan? especially those with influence? Were the available resources made ready? 

What is striking, to a former military planner, is that whilst the analysts seem to be well catered for within the POP community, planners are not. SARA is an analytical framework for POP. It is not a plans or an implementation framework, if it was an implementation framework then there would be processes for deciding how to judge and select responses, how and when to synchronise events, resource planning stages, questions regarding control measures (not comparison studies, but deconfliction in space and time), methods to test the plan and communication strategies. These are all valuable and necessary elements of a plan, but they seem to be missing from the POP literature.  

\parencite{hinkle2020problem} cites many implementation issues with third parties to the process, but if these have not been carefully managed into the process, and failure on their part understood, then the plan was never robust in the first place. That POP is not seen as mainstream policing is almost certain to hamper the process, the necessity to deal with immediate problems now is very hard to combat. Strong leadership, good plans and a cast iron belief that slower burning strategies will eventually pay-off  are required for implementation to be conducted effectively. 

 
 \subsection{Weakness 3 - The requirements for evaluation}



Evaluations are rarely sexy, and sometimes the resource or the impetus to conduct the evaluations vanishes as the project advances. Pet projects that aren't producing the results can fade away, while other project outcomes are so obvious that an official evaluation isn't needed. Indeed it is not in the culture of many organisations, especially the police to systematically review their performance\parencite{goldstein1990}. Proper evaluation will make the whole process much more efficient, analytical products and expertise will have lasting benefits if the real mechanism of change and benefit are realised. The jump in robust studies between the two Campbell reviews is encouraging, and the focus on evaluation in the POP awards in the UK and US will push it further along.  However it is worth noting at this stage that just under a third of submissions for a UK POP award\footnote{Tilley Awards} did not include any formal evaluation \parencite{POPUCL}. Again in this area, as above,  the necessity for skilled practitioners to do the work and leaders to allow them to do the work (or even mandate them) seem to be keep elements for unlocking progress.  


Three areas of weakness have been presented above, conduct, delivery and evaluation of POP. However in each of those there can be identified cross cutting themes that if addressed would lead to better POP outcomes. Two of these themes are access to information and analysis \parencite{POPUCL}. Access to information and analysis are also two areas were computational power and modern information systems can alleviate the workload. Information retrieval from document bases can be made much more efficient (if in doubt think about google and other web search engines \parencite{manning2008introduction}) and the information within those documents can now either be automatically extracted or summarised \parencite{kumar2011natural}. That is to say that the underlying conditions that are creating these problem for POP can be at least partially addressed by leveraging technological improvements.

\section{POP Conclusions}


It has be shown that POP is a successful practice for reducing crime and broader harms to the public, additionally it has been shown that this success comes down to understanding a problem in great detail, utilising all available information to then form a very specific response. Although the process involves the wider community the practice is currently led and used almost exclusively by police officers using police data whilst relying on their (scarce) civilian analysts for support. POP is successful, but it is not a quick swap for traditional policing, the resources POP requires are not readily found in police forces in sufficient quantity and the culture of the organisation is not geared behind it for success.  

In his article \textcite{whitherpop}  lays down the gauntlet for the next round of improvements for POP saying that \say{The Research and development agenda for POP is now that of improving its efficiency, reliability in producing the intended outcomes.}  This is where the practical focus of this project will lie. As we have seen above analysts and access to information is key for a successful POP implementation, but is often hampered by a lack of resource, either in the form of analytical support or resource to review disparate information. This piece of research will attempt to leverage new, but existing, machine learning techniques to partially automate the extraction of information from police crime notes, in the belief that by doing so the analytical burden for the scanning and analysis phases of the SARA cycle can be alleviated to some degree.