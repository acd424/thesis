\chapter{Introduction}

\section{Motivation}
This research arrives at the intersection of a well-established crime reduction methodology, Problem-Oriented Policing (POP), and a growing field in artificial intelligence, Natural language Processing (NLP), that is increasingly making it easier to draw information from unstructured data. 

POP is a method of policing, first introduced in 1979 by Herman Goldstein \parencite{gold79} . POP is  a policing model to replace the traditional policing model that focusses on responding to single incidents as they occur. By contrast POP seeks to prevent problems from reoccurring through analysing how they occurred in the first place, then intervening in that generation process. In this regard an essential element for conducting POP is understanding the conditions that allowed the problem to occur. Crimes are a sub-set of the problems that police forces face, albeit a large and important subset. POP's focus is on problems not just crimes. The POP terminology can stray to just focussing on crimes. Where the terminology does stray, this is normally without loss of generality of the effect of the POP on all problems encountered by the police.

POP seeks to tackle problems and defines problems as \say{A cluster of similar incidents, whether crimes or acts of disorder, that the police are expected to handle} \parencite{popchap11}. Of immediate interest from this definition is that problems should have similar incidents, and that is not just related to the outcomes but the processes and external factors that lead up to, during and after the incident. Most of the analytical effort required in POP is expended in scanning for then grouping similar crimes, and then fully analysing incidents to identify similar factors, processes or mechanisms influencing the incident occurrence. It is with these two analytical process that the research will focus.

This intersection between POP and NLP is important, as although POP works \parencite{hinkle2020problem}, it is necessary yet seemingly difficult to follow the POP framework correctly. Thus although POP has shown benefits, it has not realised its full potential \parencite{POPUCL}. The impediment to POP that this body of research hopes to reduce is the analytical burden necessary to understand the specificity of the problem or problems at hand. POP works best by attacking the mechanisms of the problems so that the opportunities to commit the crime (or other non-criminal activity) is significantly reduced \parencite{clarke2003becoming}. This is achieved through understanding the causes and the mechanism of the problems then working on ameliorating strategies. Understanding problems so that they can be attacked, but also grouping problems so that solutions can be used efficiently is a key component of POP. However, studies have shown that the analytical power and the data required to do this efficiently is difficult for police forces to muster then coordinate \parencite{sidebottom2020implementing}. 

Although police forces are mandated to record all crime \parencite{home2020crime} the bulk of the information that is recorded about crime is contained in textual data, some of this will be in police generated crime notes, witness statements or forensic reports. Accessing this information is largely completed manually \parencite{goldstein1990}, and as such it is often a long and laborious task, and given the resource pressures, the work has to be completed selectively \parencite{rogerson2016utility}. Unlocking access to this information would enable analysts and officers practicable access to a much wider source of information with which to do their job. An important sub-set of this textual data is the Modus Operandi notes that are mandated to accompany every recorded crime. Two sources of text data are introduced next as a potential sources of information.   
 
Modus Operandi (MO) data, are relatively short sections of text of around three to eleven  sentences that describe what is initially known about the crime.  The text is generally limited to the knowledge that can be gathered by the initial responding officers, from their provisional review of the crime scene and any victim or witness statements. Further investigations,  for instance by detectives or forensics are held in the case reports and are not detailed in the MO data. As such they offer a concise but limited view of the crime. Along side the MO data more typical crime data is recorded in a structured way, fields such as time, date, location, crime classification, and victim characteristics are often held. These more structured statistics have been exploited to a much greater extent  than the unstructured MO data. See \textcite{mapchap10, ratcliffe1998aoristic, braga2014effects, weisel2016analyzing}  for a selection of methods.

A second source used in this thesis is police incident logs. Police incident logs are generally generated by a call operator who responds to emergency and non-emergency calls from the public. Typically they record the details of an incident as it is in progress. More recently incident logs can also encompass reports from the public that have been logged electronically. Either through email or online forms. Police incident logs differ from MO data in two important ways. Firstly they are not generally edited, they provide a few over time rather than a single post-hoc view. Secondly they cover both crime and non-crime incidents, so they have a much broader reach than MO data. The two data types are discussed more extensively in Chapter 8.

Recent advances in NLP where the basis of models has moved from a more logical and rules based approach to a more probabilistic approach has allowed much more powerful models to be brought to bare on free-text problems \parencite{kumar2011natural}. Improvements in processing power and the availability of data have also pushed the boundaries of the state-of-the-art (SOTA) models. The improvements in NLP have led to suites of generic open source tools being developed  \parencite{manning2014stanford, benoit2018quanteda, loper2002nltk}. These toolkits are designed so that they can be reused on different sets of natural language texts to solve similar problems such as classification or question and answering, without the need to build whole models from scratch each time a problem is encountered.

Pre-trained language models (PTMs) are an import class of these generic NLP tools. A useful analogy for understanding PTMs is a university student embarking on their first graduate job. The training to be successful at the job will be in two parts. Firstly they have had their broad formal education that has culminated in a university education. They know lots of things, they understand broad concepts, but it has taken many years and lots of effort to get them to that point. When they reach their new job they will need additional specific job training. Training specific to the problems they need to solve for that particular role. The (former) student needs additional domain knowledge, that will build on top of the broad concepts that they already understand. But this additional knowledge is normally quicker to impart because of their already broad understanding of the underlying concepts.  

PTMs work similarly to this. Using a PTM is like employing a graduate for the first time. The model already has some understanding, or knowledge, of the problem. In this case the problem is what English words mean. But the model does not understand our specific problem well. Therefore we have to give the model some on the job training before we release it for work. Previously you could not \emph{employ a graduate} you had to do all of the model training yourself. But now, with the introduction of PTMs,  you can skip most of the training and start with a model that already has a broad understanding of the problem. This means that a huge amount of complexity and effort in using NLP models has been removed from the end user. 

In this research the focus is broadly on investigating the use of these PTMs with police free text data. Whilst they have been found to work well in other domains, they have yet to be tested on free-text generated by the police on problems that are important to the police. Can these PTMs be leveraged by the police and therefore take advantage of the lower barriers to use? That is the fundamental question of this thesis.



\section{Research Questions}

The research question and its supporting objectives are stated here with a brief explanation as a handrail for the reader to guide them through the next few sections. The research questions will be examined more thoroughly in relation to the literature outlined in the rest of the document in Chapter 7. The main research question is:

s

In this thesis extracting information will focus on automatically classifying texts to understand if an event did or did not happen. For example Burglary MO texts will be classified to understand if the burglar used force to enter a property or not.

The research supporting objectives are:

\begin{enumerate}
\item {\bf Identify the extent of NLP usage with police data.} This will be largely conducted in the literature survey which will be the focus of Chapter 6.

\item {\bf Evaluate how effective PTMs are with MO data.} PTMs will be formally introduced and explained in Chapter 5. MO data will be introduced in Chapter 8. Study 1 will investigate the use of PTMs to classify MO text data.

\item {\bf Evaluate how effective PTMs are with Police Incident data.} As mentioned police incident data is another source of information on problems the police face. Using PTMs to classify police incident logs will be investigated in Study 2 

\item {\bf Evaluate how effective Active Learning is with police data.}  Active Learning is a method to reduce the amount of data that PTMs need to learn. It has been found to work with other data types but its effectiveness with police data is unknown. Active Learning is introduced in Chapter 4 and studied in study 1.

\item {\bf Identify which parts of the POP process might be best supported by the use of PTMs.} The POP process will be explained fully in Chapter 3. It is likely that different parts of the process will find differing uses and utility for PTMs. Lesson for POP will be drawn from both studies and outlined in Part 3. 

\item {\bf Identify implementation barriers for PTMs.} Any new process is likely to have implementation barriers, these are important to identify so that they can be minimised. Discussed in Part 3. 

\end{enumerate}



\section{Thesis Structure} This thesis is made of three parts. Part 1 focusses on the introduction and background to the research. Part 2 is the studies exploring the use of NLP with police free text data. Part 3 draws on the studies of Part 2 and explores the implications for POP. 


Part 1 will begin with an introduction, which you have just read, that will set out the main ideas of the research. The next chapters will then go on to explain the theoretical underpinnings of POP, namely routine activity theory and situational crime theory. POP will then be discussed in more detail, identifying key problems with wide spread usage. After POP has been explained the focus will switch to the more technical aspects of the research. Firstly machine learning (ML) will be introduced. Next ML with free-text data, namely Natural language processing (NLP) , will be explored including the theory and use behind pre-trained language models (PTMs). The penultimate chapter will draw these two (POP and NLP) separate areas together by conducting a literature survey of the use of NLP with police generated free-text data. 

Part 2 will focus on the two main study areas. These study areas are delineated by the type of police free text data used. Both studies are focussed on the utility of PTMs with police generated free text. Study 1 uses Modus Operandi (MO) data. Study 2 uses police incident log data. Study 1 is split into three parts. The first part, study 1a, investigates the classification of MO texts in one police force area (known as PF1). Study 1b investigates the efficiency of Active Learning, using the data and models from study 1a. Study 1c replicates and extends study 1a using data from a separate police force (PF2). Study 2 only has one part, and is focussed on classifying police incident logs in a single police force (PF2). There is a table at the end of Chapter 7 (\ref{tab:study}) that captures this detail.

The final part of the thesis, Part 3, summarises the lessons and implications from the studies in Part 2 inlight of the conclusions from Part 1. It does so in two chapters. The first chapter discusses the implication of NLP usage for POP. In particular how and where PTMs might be used to alleviate the analytical burden. The second chapter takes a broader look at potential future research directions of PTMs with police free-text data.




