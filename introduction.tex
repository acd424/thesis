\chapter{Introduction}

This transfer paper provides an outline of the research to be conducted. Firstly, it will start with a brief explanation for the motivation for the research and what data is likely to be available. After a brief outline of the research questions there will be an explanation of Problem-oriented policing (POP) an overview of its effectiveness and problems with its implementation. POP will provide the context for the research. Following from the POP section there will then be a brief overview of machine learning and natural language processing (NLP), which will include dominant techniques in the field, a general framework and drawbacks to their use. NLP will be the set of tools through which the research aims to suggest efficiencies for the improvement of POP. Bringing these two sections together will be a literature survey into the use of NLP with Police generated textual data, which will explain the current state of research in the area. The research questions that the thesis will address will then be developed in light of the literature review and an outline research plan will be proposed.

\section{Motivation for the Research} 

This research arrives at the intersection of a well-established crime reduction methodology, Problem-Oriented Policing, and a growing field in artificial intelligence, Natural language Processing, that is increasingly making it easier to draw information from unstructured data. 

POP is a method of policing, first introduced in 1979 by Herman Goldstein \parencite{gold79} as a policing model to replace the traditional policing model that focusses on responding to single incidents that have already occurred. By contrast POP seeks to prevent  problems from reoccurring through analysing how they occurred then intervening in that generation process to prevent reoccurrence. In this regard an essential element for conducting POP is understanding the conditions that allowed the problem to occur. Crimes are a sub-set of the problems that police forces face, albeit a large and important subset. POP's focus is on problems, but the terminology can stray to just focussing on crimes, where the terminology does stray to just focussing on crime this is normally without loss of generality of the effect of the POP processes to all problems encountered by the police, not just crimes. 

POP seeks to tackle problems and defines problems as \say{A cluster of similar incidents, whether crimes or acts of disorder, that the police are expected to handle} \parencite{popchap11}. Of immediate interest from this definition is that problems should have similar incidents, and that is not just related to the outcomes but the processes and external factors that lead up to, during and after the incident. Most of the analytical effort required in POP is expended in scanning for then grouping similar crimes, and then fully analysing incidents to identify similar factors, processes or mechanisms influencing the incident occurrence. It is with these two analytical process that the research will focus.

This intersection between POP and NLP is important, as although POP works \parencite{hinkle2020problem}, it is necessary yet seemingly difficult to follow the POP framework correctly, thus although POP has shown benefits, it has not realised its full potential \parencite{POPUCL}. The impediment to POP that this body of research hopes to reduce is the analytical burden necessary to understand the specificity of the problem or problems at hand. POP works best by attacking the mechanisms of the problems so that the opportunities to commit the crime (or other non-criminal activity) is significantly reduced \parencite{clarke2003becoming}. This is achieved through understanding the causes and the mechanism of the problems then working on ameliorating strategies. Understanding problems so that they can be attacked, but also grouping problems so that solutions can be used efficiently is a key component of POP, but studies have shown that the analytical power and the data required to do this efficiently is difficult for police forces to muster then coordinate \parencite{sidebottom2020implementing}. 

Although police forces are mandated to record all crime \parencite{home2020crime} the bulk of the information that is recorded about crime is contained in textual data, some of this will be in police generated crime notes, witness statements or forensic reports. Accessing this information is largely completed manually \parencite{goldstein1990}, and as such it is often a long and laborious task, and given the resource pressures, the work has to be completed selectively \parencite{rogerson2016utility}. Unlocking access to this information would enable analysts and officers practicable access to a much wider source of information with which to do their job. An important sub-set of this textual data is the Modus Operandi notes that are mandated to accompany every recorded crime. This text data is introduced next as a potential source of information.   
 
Modus Operandi (MO) data, sometimes referred to as crime notes, are relatively short sections of text of around three to eleven  sentences that describe what is initially known about the crime.  The text is generally limited to the knowledge that can be gathered by the initial responding officers, from their provisional review of the crime scene and any victim or witness statements. Further investigations,  for instance by detectives or forensics are held in the case reports and are not detailed in the MO data. As such they offer a concise but limited view of the crime. Along side the MO data more typical crime data is recorded in a structured way, fields such as time, date, location, crime classification, and victim characteristics are often held. These more structured statistics have been exploited to a much greater extent  than the unstructured MO data. See \textcite{mapchap10, ratcliffe1998aoristic, braga2014effects, weisel2016analyzing}  for a selection of methods.

Recent advances in NLP where the basis of models has moved from a more logical and rules based approach to a more probabilistic approach has allowed much more powerful models to be brought to bare on free-text problems \parencite{kumar2011natural}. Improvements in processing power and the availability of data have also pushed the boundaries of the state-of-the-art (SOTA) models. The improvements in NLP have led to suites of generic open source tools being developed  \parencite{manning2014stanford, benoit2018quanteda, loper2002nltk}. These toolkits are designed so that they can be reused on different sets of natural language texts to solve similar problems such as classification or question and answering, without the need to build whole models from scratch each time a problem is encountered.

In this research the focus is broadly on investigating the use of Natural Language Processing techniques to automatically extract information from free text data, more specifically it will be centred on extracting useful information from Police Modus Operandi textual data to lower the analytical burden on analysts to enable more efficient implementations of POP, with the ultimate aim of reducing crime.



\section{Research Questions}

The research question and its sub-questions are stated here with a brief explanation as a handrail for the reader to guide them through the next few sections. The research questions will be examined more thoroughly in relation to the literature outlined in the rest of the document in Chapter 7.

\textbf{Can NLP techniques be used to extract crime prevention information from modus operandi textual data, and if so what practical applications does it have?}

Chapter 3 will be an introduction to POP and one of the main focusses of this chapter will be demonstrating that although POP is effective, it has a high analytical burden. The strength of POP lies in fully understanding why and how a group of problems has occurred, as developed in Chapter 2, so that interventions can then be made to disrupt this process. 

Understanding what processes generating the conditions for problems are common across problems allows a bespoke yet efficient response to be implemented. However the high analytical burden is an impediment to the efficient implementation of POP, generally this burden centres on the ability to access then analyse sufficient information about the single incidents that make a problem. As a lot of the information required is contained within free text police data, this research aims to show that natural language processing techniques will be able to provide the required information in a more efficient manner than the current largely manual processes, and by doing so will improve the efficiency of POP.

The research sub questions are:

\begin{itemize}
\item {\bf How effective are established NLP models in analysing police generated modus operandi free text?} Chapter 5 will introduce NLP and show that there are many parts of the NLP process that have open source state of the art models that have been successfully utilised on existing data sets. However these models are trained on data that is likely to be different to police free text data and this means that they may not produce optimal results on police data. These models need to be tested and results analysed too quantify what additional work is require to optimise the use of NLP models on police generated free-text data.

\item {\bf What are the challenges associated with preparing modus operandi data for analyses by existing NLP methods, and how might they be overcome?}  The more powerful sub-class of machine learning algorithms for extracting information is supervised learning. As Chapter 4 will demonstrate in order to build models with supervised learning techniques the models require training data that has already been annotated with the desired results. This annotation of data, called labelling,  can be resource intensive and therefore a potential barrier to practical utility of NLP methods.  This element of the research will seek to identify which labelling strategies are the most efficient for the implementation of supervised machine learning techniques in a POP context. 

\item {\bf Which parts of the POP process might be best supported by the use of these techniques?}  Extracting information and evaluating the process on a technical level against labelled data is the traditional method for judging the accuracy of a NLP project. However this research will attempt to extrinsically validate that accuracy against the utility that POP practitioners may expect to gain from possession of that information. Although not fixed, the research expects to judge utility in two separate parts of the POP process, the first will be to assist the groupings of incidents into problems in the early stages of a POP process and the second will be to understand how crime/problem processes have changed in-light of an intervention, as required in the the evaluation stage of POP.   

\end{itemize}

Across these sub questions the issue of practicality will also be addressed. Practicality in this case will chiefly encompass resource burden and explainability of models utilised.

\section{Published Work} Throughout this first year of study I have also widened my general criminological knowledge by partaking in two additional research projects. 

The first project was an investigation into the Age Period and Cohort effects on motor vehicle crime in the United States of America \parencite{dixon2020age}. This project allowed me to understand how crime opportunities are constructed, and more generally the construction of academic arguments and evidence.

The second project is reducing the unanticipated crime harms of Covid-19 policies.  This is an ERSC funded project to investigate the harms that may result form the changes in crime patterns as a result of the underlying changes in the crime opportunity structure. As part of this project I have been the lead author for the projects statistical bulletin series \footnote{www.covid19-crime.com}. I have also co-authored a paper highlighting the changes in crime for a single regional police force in England \parencite{halford2020crime}. This project has allowed me to extend my academic skills, but additionally it may also feed directly into this project by utilising NLP techniques to quantify and understand any of the intra-crime classification changes that may have occurred as a result of the pandemic.
