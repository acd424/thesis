\chapter{Conclusions}

\section{Introduction} This is the final chapter of the thesis. This chapter will consist of two sections. The first section will explore how the reseaerch covered in the thesis can be expanded. The  further development of the research can be through two main avenues. Firstly by improving the research conducted here and secondly by using the nlp models in ways that have not been explored in this thesis. 

The second section will be the concluding thoughts for the thesis and will summarise the research conducted in this thesis. 

\section{Future Research} Future research for utilising NLP models with police data is vast. The field of NLP is continually growing and NLP models are also becoming more capable in different ways, so the application of NLP to police data is and will continue to be a dynamic field. The future reaserch is split into two areas. Firstly there is a section on models, this section is broadly focussed on how the research within this thesis may be imroved. Though it will also have broader applicability. The following section is entitled Applications and is aimed to show how NLP research can be broadened beyond the scope of this thesis.  the applications section is a brief description on how NLP models can be used more widely to help the police more readily access the information held within their free text data. 

\subsection{Models}  This section focuses on how future research can improve upon the models produced in this research, namely the use of PTMs to classify short pieces of text.

\subsubsection{Further replication} The research here has been narrow in scope. Only one type of crime was investigated and with only three different classification typoes. Although this was partially replicated across two different police forces.  This research should be expanded to include addition crimes, different classification s and additional police forces. In particular further replication of the same use cases across different police forces would allow a much greater understanding of how well the models can be reused in different police forces. If models can be reused across police forces then this will reduce the labelling burden as a single model can be produced rather than forty-three separate models (one for each force in the UK). 

\subsubsection{Type} The model used in this research was BERT. Since BERT was built there have be more PTMs produced and made available for free use.  Each of these PTMs has its own characteristics, capabilities and therefore linguistic areas where it excels. By experimenting with different kinds of PTMs, one can discover which one works best for specific uses cases. As an example another popular PTM is ROBERTA. ROBERTA uses a different method to define which words are being used. This difference means  that  it handles previously unseen words in a more robust way. Police data with lots of acronyms or obscure words may be better represented by this model type, and so classifications may become more accurate. Other models have larger architecture i.e. more parameters to tune. This larger architecture means that bigger models are able to represent more challenging nuances in the text and thereby give more accurate classifications.  Model types are likley to evolve and so understanding which models are most suited to the police data being used will be an ongoing process.

\subsubsection{Hyperparameter tuning} Earlier in the thesis hyperparameters were introduced. Hyperparameters are varibales in the model formulation that alter slightly how it trains. An example of a hyperparameter is the number of epochs. Epochs represent the number of times the whole training set is used to train the model. In this research three epochs was used. That means that the model saw each piece of training data on three separate occasions. Tuning hyperparameters involves adjusting their values in order to optimise the model's performance. This can be a time-consuming process, but it is important because the right combination of hyperparameters can improve the model's performance. Hyperparameter tuning was not conducted in this research because the idea was to use a simplified process for classifying the texts. A simple process that could be easily implemented in a police force.  The results from this thesis indicate that the default hyperpareameters, i.e untuned hyperparameters, produced satisfactory models. However hyperparametreer tuning could lead to either more accurate classifications or  a lesser requirement for labelled data. Either way adding hyperparameter tuning may lead to an improvement in model performance and so would be a good avenue for further research. 

\subsubsection{Outcome weighting} In this research the getting the classification wrong was weighted equally with getting the classification correct, and so the models were trained to reduced the amount of incorrect classifications. However it may be the case, as explored in the conclusion of study 2, that getiing a classification wrong is not equall in all instances. For instance it might be that missing a burglary where a car was stolen is worse than misclassifying a burglary where a car was not stolen. To put this into sharper focus an alternate problem might be trying to find vulnerable victims, missing a vulnerable victim may be more costly than misclassifying non-vulnerable victims. 

To overcome this problem a technique called outcome weighting is used in machine learning to adjust the importance of different outcomes in a classification problem. In reference to the theoretical problem introduced earlier missing a vulnerable victim might be classed as twice as costly as classifying a non-vulnerable victim as vulnerable. This cost function will have to be built with the end user so that their understanding of the problem, and the costs of misclassification can be coded into the model training. Typically this weighting can be either encoded into the loss function so that the model training is changed or the model outputs can be used in a more sophisticated way to deliver the desired outcome.  


\subsubsection{Vocabulary} BERT has a set amount of words that it recognises. This is called the models vocabulary. The benefit of a word being in the vocabulary is that the word will have a more defined numerical representation. If a word is not in the vocabulary then it is broken down into word pieces until it is recognised, in extremis some words can be classed as unknown. Breaking a word into word pieces can destroy some of the meaning of that word as it is not represented as as single entity. In text where there are a lot of out of vocabulary words the meaning of those words may  not be represented well and therefore the classification models may not be that accurate. 

There are two ways to overcome this problem. Firstly the vocabulary of BERT can be extended so that the it contains other words. The most popular unknown words can be added to the vocabulary thus preventing the word form being broken down. Secondly the unknown words can be changed to  a word or words that is already within the BERT vocabulary.  For instance in say{untidy} was not recognised by BERT and so could be replaced with say{messy} or say{not tidy} which are both recognised by BERT. 

Overall, making the text and BERT vocabularies more similar can help to improve PTMs performance on specific tasks by increasing the models understanding of the domain in question.

\subsubsection{Pre-train} As mentioned previously there are two parts to utilising a BERT model. There is the pre-training element - which isresource intensive and  give s the model a general understanding of language and then there is pre-training which is  conducted for each specific task. The pre-training was not completed for this research, but in other domains where they have had access to sufficient data, they have conducted pre-training. Where this pre-training has been conducted new variations of BERT have been built. For instance Legal-BERT has been built to understand legal documents \parencite{legal_bert}. Another variation trained on medical data is med-BERT. 

Therefore an interesting avenue of research would be to pre-train a BERT model on police data, perhaps exclusively MO data from across several different forces, to produce an MO-BERT. Given the success in other domains this new model is likely to perform better at classifying MO texts than the regular BERT. This approach may therefore save time and resources when fine-tuning for each additional task as it has a better understanding of the domain specific language from the outset.

This section has demonstrated that there are a number of interesting avenues for additional research to enhance the classification work outlined in this thesis. The next section takes this further by exploring what other NLP techniques, beyond classification of text passages, can be used to enhance problem-oriented policing. 

\subsection{Applications}

\subsubsection{Question and Answer} Question answering (Q\&A) is a NLP task that involves using a PTM to answer questions posed in natural language, given a text which contains the answer. This is different to chat style AI where the PTM uses only pre-defined knowledge to answer the question. So for example in our case a police officer may have one or more documents. These documents will then be submitted to the PTM with a question. Using only the text available in the documents the PTM will then generate a response. Q\&A systems can be designed to answer a wide range of questions, including factual questions, definitions, and queries about people. However they have not been trialled against crime documents and so their performance in this area is unknown. This may be more useful when the crime is of low volume or a specific responce is needed rather than a binary classification.

\subsubsection{Named Entity Recognition} Named Entity Recognition (NER) is another NLP task that is based on classification. In this instance rather than classify a passage of text it classifies every word within that text. The typical task for this type of task is to extract organisations, people and places for m a passage of text. The PTM in this instance will label each word within the text as either nothing, a person and organisation or a place. For example, in the sentence "Boris Johnson was born in New York on 19 June 1964" the named entities are \say{Boris Johnson}, \say{New York} and \say{19 June 1964}. Where words are positively identified these can be extracted from the text. The PTM uses both the word itself but also the context of the word to output the classification. Therefore names or places that were never in the training material can still be correctly extracted because they will be used in a similar context. In a policing context the words of interest may not be people or places, but may for instance be the weapon used in an assault.


\subsubsection{Summarisation} Text summarization involves producing a short summary from a longer document or a collection of documents. The idea is to keep the most important information from the original documents so that the summarisation can be read in isolation. This task is well understood in the NLP domain but is hard to generalise across different language domains because 1) it is inherently hard to quantify what is a good summary and 2) the importance of different facts across domains. As the quality of a summary is difficult to quantify it means that there has to be more human intervention in the modelling process which makes it more resource intensive than other NLP tasks. The importance for the police domain is that reviewing cases would be much quicker and easier with a case summary. In places these are already produced by an Officer 


\subsection{Data}
\subsubsection{Different types} The types of text data that police departments may have can vary depending on the specific needs and goals of the department. Some examples of text data that police departments may have include:

Police reports: Police reports are written documents that describe the details of an incident or crime that has been reported to the police. These reports may include information about the location, time, and nature of the incident, as well as the names and contact information of witnesses and suspects.
Arrest and booking records: Police departments may have records of arrests and bookings, which include information about individuals who have been taken into custody and the charges that have been filed against them.
Interrogation transcripts: Police departments may have transcripts of interrogations that have been conducted as part of an investigation. These transcripts can include the questions asked and the responses given by the suspects or witnesses.
Communication logs: Police departments may have logs of communication between officers, such as radio transcripts or email exchanges.
Social media posts: Police departments may also have access to social media posts that are relevant to an investigation, such as posts made by suspects or witnesses.

\subsubsection{Languages} All though all of the research here has been with texts in the english language similar models do exist for other languages. In addition translation models also make it possible to translate non-english text into english in order to use english models. Further research utilising the non-english models to prove that similar tasks are possible in non-english languages would also be useful to make the approaches explored here more widespread. 

\subsubsection{Bias} The results from the bias explorations here are encouraging, but are only reflective of one portion of the data journey from creation to model output. Particularly the research presented here focussed on algorythmic bias. To the author is knowledge there is little understanding of the biases effecting the completeness of the details in a textual crime record. This part of the data journey is worthy of more research to understand potential biases that could be inherent to the data from this route.


\section{Implications for Police Agencies}

\subsection{training, public perception, hardware/software, centralised - decentralised capability}


\section{Concluding Remarks}






