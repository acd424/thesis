\chapter{Conclusions}

\section{Introduction} This is the final chapter of the thesis 

\section{Future Research}




\subsection{Models} 

\subsubsection{Further replication} Transfer learning
\subsubsection{Type} There are many different types of language models available, each with its own unique characteristics and capabilities. By experimenting with different kinds of language models, you can discover which ones work best for your specific needs and goals. This can involve training and testing different models on different datasets, comparing their performance, and fine-tuning their parameters to optimize their performance. Additionally, you can try using different techniques and approaches to pre-process and prepare your data, such as tokenization, stemming, and lemmatization, to see how they impact the performance of your language models.
\subsubsection{Parameter tuning} Tuning hyperparameters involves adjusting their values in order to optimize the model's performance on a given task. This can be a time-consuming process, but it is important because the right combination of hyperparameters can significantly improve the model's performance. On the other hand, using inappropriate hyperparameters can result in a poorly performing model.

There are several techniques that can be used to tune hyperparameters, such as grid search, random search, and Bayesian optimization. These techniques involve trying out different combinations of hyperparameter values and evaluating the model's performance on a validation dataset in order to determine the optimal set of hyperparameters.

In summary, hyperparameter tuning is an important step in the process of building and training a machine learning model, including a language model, and can have a significant impact on the model's performance.

\subsubsection{Outcome weighting}
 Outcome weighting is a technique that is used in machine learning to adjust the importance of different outcomes or classes in a classification problem. In a classification problem, the goal is to predict the class label of a given input data point. For example, in a binary classification problem with two classes, the goal is to predict whether an input belongs to class 0 or class 1.

Outcome weighting can be used to adjust the importance of correctly predicting one class over the other. For example, if correctly predicting class 1 is more important than correctly predicting class 0, the model can be trained with higher weights for class 1, which will make the model more sensitive to errors in predicting class 1 and less sensitive to errors in predicting class 0.

There are several ways to apply outcome weighting in machine learning. One common approach is to use weighting in the loss function, which is a measure of how well the model is performing. For example, in a binary classification problem with class weights w0 and w1, the loss function could be defined as:

loss = w0 * loss_for_class_0 + w1 * loss_for_class_1

Another approach is to use weighting in the evaluation metrics, such as precision, recall, and F1 score. In this case, the evaluation metric would be calculated separately for each class and the weights would be applied to the final score.

Outcome weighting can be useful in situations where the classes are imbalanced, meaning that there is a disproportionate number of examples belonging to one class compared to the other. In such cases, the model may tend to predict the majority class more often, leading to poor performance on the minority class. Outcome weighting can help to balance the importance of the different classes and improve the overall performance of the model.

\subsubsection{Vocabulary} One reason to add vocabulary words to BERT is to improve the model's ability to understand and process a specific domain or topic. For example, if you are using BERT for sentiment analysis of customer reviews for a particular product, you may want to add vocabulary related to that product to the model. This can include product-specific terms, technical terms, and common phrases that are used in the reviews. By adding these words to the model's vocabulary, you can improve its ability to accurately classify the sentiment of the reviews.

Another reason to add vocabulary words to BERT is to improve its performance on a specific language or dialect. BERT is trained on a large dataset of generic English text, but it may not perform well on text written in a specific dialect or language that uses unique vocabulary or grammatical structures. In such cases, adding vocabulary words specific to that dialect or language can help to improve the model's performance.

Overall, adding vocabulary words to BERT can help to improve its performance on specific tasks and languages by increasing its understanding of the domain or language in question.

\subsubsection{pre-train} One of the main benefits of pre-training a BERT model is that it can learn general language representations that are useful for a wide range of tasks. BERT is trained on a large dataset of generic English text and is able to learn general language patterns and structures that can be useful for many different tasks. By pre-training a BERT model, you can leverage these general language representations and use them as a starting point for fine-tuning the model on a specific task. This can save time and resources compared to training a model from scratch, and can also lead to better performance on the task.

\subsection{Applications}

\subsubsection{Question and Answer} Question answering (Q&A) is a natural language processing (NLP) task that involves using a computer to automatically answer questions posed in natural language. Q&A systems can be designed to answer a wide range of questions, including factual questions, definitions, and queries about people, places, or events.

may help where the amount of data is lower per problem.

\subsubsection{Named Entity Recognition} Named Entity Recognition (NER) is a natural language processing (NLP) task that involves identifying and classifying named entities in a text. Named entities are specific people, organizations, locations, dates, and other real-world objects that are mentioned in the text.

For example, in the sentence "Barack Obama was born in Hawaii on August 4, 1961," the named entities are "Barack Obama," "Hawaii," and "August 4, 1961." NER systems are designed to identify these named entities and classify them into predefined categories, such as person, location, and date.

\subsubsection{Summarisation} Text summarization is a natural language processing (NLP) task that involves generating a shorter version of a text document while preserving the most important information and main ideas. The goal of text summarization is to create a summary that is concise and coherent, and that captures the essence of the original text.

\subsubsection{Clustering} Clustering is a technique in natural language processing (NLP) that involves grouping similar data points together into clusters. Clustering can be useful in a variety of NLP tasks, including text classification, topic modeling, and document organization.

Some benefits of clustering in NLP include:

Reducing the complexity of data: Clustering can help to simplify large datasets by grouping similar data points together, making it easier to understand and analyze the data.
Identifying patterns and trends: Clustering can help to uncover patterns and trends in the data that may not be immediately apparent. This can be useful for understanding the characteristics of different groups and for making informed decisions.
Improving the efficiency of algorithms: Clustering can be used to improve the efficiency of machine learning algorithms by reducing the size of the dataset that the algorithm needs to process.
Facilitating visualizations: Clustering can help to create visualizations of the data that are easier to understand and interpret.
Overall, clustering is a useful technique in NLP that can help to improve the efficiency and effectiveness of a wide range of tasks.


\subsection{Data}
\subsubsection{Different types} The types of text data that police departments may have can vary depending on the specific needs and goals of the department. Some examples of text data that police departments may have include:

Police reports: Police reports are written documents that describe the details of an incident or crime that has been reported to the police. These reports may include information about the location, time, and nature of the incident, as well as the names and contact information of witnesses and suspects.
Arrest and booking records: Police departments may have records of arrests and bookings, which include information about individuals who have been taken into custody and the charges that have been filed against them.
Interrogation transcripts: Police departments may have transcripts of interrogations that have been conducted as part of an investigation. These transcripts can include the questions asked and the responses given by the suspects or witnesses.
Communication logs: Police departments may have logs of communication between officers, such as radio transcripts or email exchanges.
Social media posts: Police departments may also have access to social media posts that are relevant to an investigation, such as posts made by suspects or witnesses.
\subsubsection{Languages} Multi-lingual models or translation.

\subsubsection{Bias} More research into the data biases - particulalry the generation of text data after the crime has been reported.



\section{Implications for Police Agencies}

\subsection{training, public perception, hardware/software, centralised - decentralised capability}


\section{Concluding Remarks}






