\chapter{Conclusions}

\section{Introduction} This is the final chapter of the thesis. This chapter will consist of two sections. The first section will explore how the reseaerch covered in the thesis can be expanded. The  further development of the research can be through two main avenues. Firstly by improving the research conducted here and secondly by using the nlp models in ways that have not been explored in this thesis. 

The second section will be the concluding thoughts for the thesis and will summarise the research conducted in this thesis. 

\section{Future Research} Future research for utilising NLP models with police data is vast. The field of NLP is continually growing and NLP models are also becoming more capable in different ways, so the application of NLP to police data is and will continue to be a dynamic field. The future reaserch is split into two areas. Firstly there is a section on models, this section is broadly focussed on how the research within this thesis may be imroved. Though it will also have broader applicability. The following section is entitled Applications and is aimed to show how NLP research can be broadened beyond the scope of this thesis.  the applications section is a brief description on how NLP models can be used more widely to help the police more readily access the information held within their free text data. 

\subsection{Models}  This section focuses on how future research can improve upon the models produced in this research, namely the use of PTMs to classify short pieces of text.

\subsubsection{Further replication} The research here has been narrow in scope. Only one type of crime was investigated and with only three different classification typoes. Although this was partially replicated across two different police forces.  This research should be expanded to include addition crimes, different classification s and additional police forces. In particular further replication of the same use cases across different police forces would allow a much greater understanding of how well the models can be reused in different police forces. If models can be reused across police forces then this will reduce the labelling burden as a single model can be produced rather than forty-three separate models (one for each force in the UK). 

\subsubsection{Type} The model used in this research was BERT. Since BERT was built there have be more PTMs produced and made available for free use.  Each of these PTMs has its own characteristics, capabilities and therefore linguistic areas where it excels. By experimenting with different kinds of PTMs, one can discover which one works best for specific uses cases. As an example another popular PTM is ROBERTA. ROBERTA uses a different method to define which words are being used. This difference means  that  it handles previously unseen words in a more robust way. Police data with lots of acronyms or obscure words may be better represented by this model type, and so classifications may become more accurate. Other models have larger architecture i.e. more parameters to tune. This larger architecture means that bigger models are able to represent more challenging nuances in the text and thereby give more accurate classifications.  Model types are likley to evolve and so understanding which models are most suited to the police data being used will be an ongoing process.

\subsubsection{Hyperparameter tuning} Earlier in the thesis hyperparameters were intorduced. Hyperparameters are varibles in the model forumluation that alter slightly how it trains. An example of a hyperparameter is th enumber of epochs. Epochs represenet the number of times the whole training set is used to train the model. In this research three epochs was used. That means that the model saw each piece of trainin gdata on three seperate occaisions. Tuning hyperparameters involves adjusting their values in order to optimize the model's performance. This can be a time-consuming process, but it is important because the right combination of hyperparameters can improve the model's performance. Hyperparameter tuning was not conducted in this research because the idea was to use a simplified process for classifying the texts. A simple process that could be easily implemented in a police force.  The results from this thesis indicate that the default hyperpareameters, i.e untuned hyperparameters, produced satifactory models. However hyperparametreer tuning could lead to either more accurate classifcations or  a lesser requiremnt for labelled data. Either way adding hyperparameter tuning may lead to an improvement in model performance and so would be a good avenue for further research. 

\subsubsection{Outcome weighting} In this research the getting the classifcation wrong was weighted equally with gettin gth eclassification correct, and so the models were trained to reduced the amount of incorrect classications. However it may be the case, as explored in the conclusion of study 2, that getiing a classification wrong is not equall in all instances. For instance it might be that missing a burglayr where a car was stolen is worse than misclassifying a burglary where a car was not stoilen. To put this into sharper focus an alternate problem migjht be trying to find vulnerable victims, missing a vulnerable victim may be more costly than misclassifying non-vulneable victims. 

To overcome this problem a technique called outcome weighting is used in machine learning to adjust the importance of different outcomes in a classification problem. In reference to the theoretical problem introduced earlier missing a vunlerable victim might bve claseed as twice as costly as classifying a non-vulneable victim as vulnerable. This cost function will have to be built with the end user so that their undersatnding of the problem, and the costs of misclassifcation can be coded into the model training. Typically this weighting can be either encoded into the loss function so that the model training is changed or the model outputs can be used in a more sophisticated way to deliver the desired outcome.  


\subsubsection{Vocabulary} BERT has a set amount of words that it recognises. This is called the models vocabulary. The benefit of a word being in the vocabulary is that the word will have a more defined numerical representation. If a word is not in the vocabulary then it is broken down into word pieces until it is recognised, in extremis some words can be classed as unknown. Breaking a word into word pieces can destroy some of the meaning of that word as it is not represented as as single entity. In text where there are a lot of out of vocabulary words the meaning of those words may go 


One reason to add vocabulary words to BERT is to improve the model's ability to understand and process a specific domain or topic. For example, if you are using BERT for sentiment analysis of customer reviews for a particular product, you may want to add vocabulary related to that product to the model. This can include product-specific terms, technical terms, and common phrases that are used in the reviews. By adding these words to the model's vocabulary, you can improve its ability to accurately classify the sentiment of the reviews.

Another reason to add vocabulary words to BERT is to improve its performance on a specific language or dialect. BERT is trained on a large dataset of generic English text, but it may not perform well on text written in a specific dialect or language that uses unique vocabulary or grammatical structures. In such cases, adding vocabulary words specific to that dialect or language can help to improve the model's performance.

Overall, adding vocabulary words to BERT can help to improve its performance on specific tasks and languages by increasing its understanding of the domain or language in question.

\subsubsection{pre-train} One of the main benefits of pre-training a BERT model is that it can learn general language representations that are useful for a wide range of tasks. BERT is trained on a large dataset of generic English text and is able to learn general language patterns and structures that can be useful for many different tasks. By pre-training a BERT model, you can leverage these general language representations and use them as a starting point for fine-tuning the model on a specific task. This can save time and resources compared to training a model from scratch, and can also lead to better performance on the task.

\subsection{Applications}

\subsubsection{Question and Answer} Question answering (Q&A) is a natural language processing (NLP) task that involves using a computer to automatically answer questions posed in natural language. Q&A systems can be designed to answer a wide range of questions, including factual questions, definitions, and queries about people, places, or events.

may help where the amount of data is lower per problem.

\subsubsection{Named Entity Recognition} Named Entity Recognition (NER) is a natural language processing (NLP) task that involves identifying and classifying named entities in a text. Named entities are specific people, organizations, locations, dates, and other real-world objects that are mentioned in the text.

For example, in the sentence "Barack Obama was born in Hawaii on August 4, 1961," the named entities are "Barack Obama," "Hawaii," and "August 4, 1961." NER systems are designed to identify these named entities and classify them into predefined categories, such as person, location, and date.

\subsubsection{Summarisation} Text summarization is a natural language processing (NLP) task that involves generating a shorter version of a text document while preserving the most important information and main ideas. The goal of text summarization is to create a summary that is concise and coherent, and that captures the essence of the original text.

\subsubsection{Clustering} Clustering is a technique in natural language processing (NLP) that involves grouping similar data points together into clusters. Clustering can be useful in a variety of NLP tasks, including text classification, topic modeling, and document organization.

Some benefits of clustering in NLP include:

Reducing the complexity of data: Clustering can help to simplify large datasets by grouping similar data points together, making it easier to understand and analyze the data.
Identifying patterns and trends: Clustering can help to uncover patterns and trends in the data that may not be immediately apparent. This can be useful for understanding the characteristics of different groups and for making informed decisions.
Improving the efficiency of algorithms: Clustering can be used to improve the efficiency of machine learning algorithms by reducing the size of the dataset that the algorithm needs to process.
Facilitating visualizations: Clustering can help to create visualizations of the data that are easier to understand and interpret.
Overall, clustering is a useful technique in NLP that can help to improve the efficiency and effectiveness of a wide range of tasks.


\subsection{Data}
\subsubsection{Different types} The types of text data that police departments may have can vary depending on the specific needs and goals of the department. Some examples of text data that police departments may have include:

Police reports: Police reports are written documents that describe the details of an incident or crime that has been reported to the police. These reports may include information about the location, time, and nature of the incident, as well as the names and contact information of witnesses and suspects.
Arrest and booking records: Police departments may have records of arrests and bookings, which include information about individuals who have been taken into custody and the charges that have been filed against them.
Interrogation transcripts: Police departments may have transcripts of interrogations that have been conducted as part of an investigation. These transcripts can include the questions asked and the responses given by the suspects or witnesses.
Communication logs: Police departments may have logs of communication between officers, such as radio transcripts or email exchanges.
Social media posts: Police departments may also have access to social media posts that are relevant to an investigation, such as posts made by suspects or witnesses.
\subsubsection{Languages} Multi-lingual models or translation.

\subsubsection{Bias} More research into the data biases - particulalry the generation of text data after the crime has been reported.



\section{Implications for Police Agencies}

\subsection{training, public perception, hardware/software, centralised - decentralised capability}


\section{Concluding Remarks}






