\chapter{Implications for POP}


\section{Introduction} This is the first chapter of the third and final part of the thesis. The aim of this part of the thesis is to draw together the lessons from the previous two parts and to explain their meaning for the future of POP and NLP. This chapter synthesises the results from the previous studies in order to meet Supporting Objective 5,  \say{Identify which parts of the POP process might be best supported by the use of PTMs.} 

The next and final chapter focuses on avenues for future research on the use of NLP with police data.

This chapter has three sections. The first section refers to the SARA framework for POP and identifies areas of the framework for the employment of PTMs. The second section concerns two additional matters that are related to the use of PTMs, namely 1) the implementation style of POP in police  forces and 2) the sharing of fine-tuned PTMs. The concluding section overviews the technical and physical barriers to the implementation of PTMs in the POP cycle.


\section{POP Applications} Part 1 explored POP. A POP framework, SARA, was introduced as part of that exploration (see Figure \ref{fig:SARA}). SARA is a four point framework, the elements of the framework are - Scanning, Analysis, Response and Assessment.  Although there is a natural order to the framework, it should be stressed that moving back and forth is encouraged because it enables the process to be refined. The following sections briefly restates the aims of the elements of the SARA framework and explores the potential applicability of PTMs.

\subsection{Scan for Problems} Scanning is the first stage of the process, and it revolves around finding and defining the problem that is to be solved. By way of reminder, a problem is a cluster of similar and related incidents that cause harm to the public and can be considered to be a police responsibility. Problems are not necessarily crimes. In fact, the ASB from Study 2 is an example of a  serious non-crime  problem. 

Generally, scans are conducted on the basis of prior information, that is, the individual who scans already has an idea about, for instance, the type of incident and the variation that they are looking to identify. In this instance, an attempt is made to confirm or rule out that variation and to find other problems with the same characteristics. Alternatively, the scan may focus on problems of unknown form and variation, such as high-harm or novel problems. This second type of scan is examined at the end of this section. 

The scan is conducted with a general idea in mind about the problem type. As an example we can use the second problem identified in Chapter 10 (10.1.1). The problem was to determine whether an outbuilding or a home had been burgled in each recorded burglary. The detective was aware of a spate of burglaries but believed that it had been the result of a high proportion of outbuilding-only burglaries. In this instance, the suitability of PTMs can be determined by answering two questions. 1) Are the data available in a structured format? 2) Is the problem large enough to justify the labelling burden? The first question is whether there are enough suitable and structured data to answer the question. Structured data are much easier to handle and analyse than unstructured data and should always be prioritised. If structured data are found, then their suitability should be tested, for example for completeness and accuracy. In the example, some structured data were available. For instance, crimes are classified as burglaries. These structured data enable the search space to be reduced but do not enable a detailed scan of the variation within burglaries. It is known that information about the variation of interest (outbuilding or not) is not accessible from the structured data. Therefore, PTMs are useful for extracting information from the unstructured data source and for presenting it in a structured manner. Accordingly, in this thesis, burglaries were classified depending on whether they only targetted an outbuilding.

The second consideration has to do with the volume of potential incidents that need to be scanned. All of the studies in this thesis show that PTMs, being a form of supervised learning, require labelled data for fine-tuning. For PTMs to be accurate when applied to the data that were used in this thesis, it was necessary to read and label between 700 and 900 MO texts. This impacted the utility of the PTMs. For the labelling exercise to be efficient, the pool of potential incidents must be sufficiently large. If the area of interest had only had 100 burglaries in the previous year, then the PTMs would not have been efficient. However, if an area, and perhaps a comparison area, have had thousands of burglaries, then it becomes more likely that the PTMs would be efficient.

PTMs are likely to be useful at the scanning stage. Their usefulness depends on there being a gap in the knowledge that is generated from the structured data and sufficiently serious potential problems that would justify the effort of using PTMs (primarily labelling costs). This is predicated on a known problem. There can be occasions on which the exact nature of a problem is not known, as alluded at the start of the section. In that case, PTMs and NLP techniques can be used, but not in the way (supervised) in which they were employed in the studies here.For unknown problems PTMs must be used unsupervised. In the unsupervised case, the machine learning algorithm clusters the data according to the variation that the PTM finds. A similar method was used by Birks et al. (2020). They clustered burglaries without using prior information about the desired themes of the clusters. This highlights a key limitation of the use of PTMs in the way that is explored in this thesis (supervised) – one must know what problem variation one is looking for in order to explore it.

In summary, PTMs can be useful for the scanning phase of the POP process because they allow additional information to be unearthed from unstructured data sources. That information can then be used to solve group problems. Once grouped, the problems need to be analysed in order to determine how they occur. This issue forms the subject matter of the next section.


\subsection{Analyse in Depth} This part of the framework entails arriving at a more complete understanding of the causes of problems. What underlying mechanisms generate a given problem? Although there are some variations between problems, it is important to identify key areas of overlap. POP practitioners must delve deeper into problems than in the scanning phase. They must gain more information about the problems in order to understand developments. This deeper analysis is likely to involve more unstructured data, and PTMs can facilitate its systematic analysis. The set of relevant data sources is likely to be expanded. Although only high-level overviews of the problem may be utilised, the analysis phase is likely to involve work with more detailed and therefore lengthier documents, such as witness statements and other police reports.

This thesis showed that as documents become longer, the ability of PTMs to analyse them efficiently becomes more limited. The structure of PTMs does not allow computations to be scaled linearly with the length of the document. Consequently, the PTM analysis of longer documents requires more computational resources. Study 2 introduced a PTM, Longformer, which is designed for longer texts. The texts in Study 2, although larger than MO texts, were not particularly long, in terms of word count, especially if compared to witness statements. The median length of the police incident logs was 166 words. Witness statements can run to several pages. With each page containing up to 500 words, they are likely to be longer than police incident logs and therefore to require more computational resources. The use of existing PTMs for texts of this length has not been studied extensively. However, one paper from the medical literature \parencite{limitations_of_transformers}  indicates that current models lose some of their effectiveness when applied to long texts (circa 2,000 words). This loss of effectiveness, coupled with the high computational costs, may mean that, at this stage of their development, PTMs are not suitable for the more detailed work that the analysis phase requires. Other NLP models may be appropriate, depending on the exact nature of the problem and the texts, but that issue is not investigated here.

In short, the analysis phase of the SARA framework is not likely to be the most appropriate for the exploitation of PTMs due to model limitations that have to do with the lengths of texts. Long texts are required in this phase of SARA because of the additional detail required for each incident.


\subsection{Respond} The response stage is about designing and implementing a response. This phase is about considering the evidence that has been amassed in the course of the two preceding stages and about designing a strategy for eliminating the conditions that cause problems to occur. Ideally, a response should not lean on enforcement activity and should account for previous solutions to similar problems. Unlike the first two stages, the third one does not entail reading similar descriptions of problems and trying to extract information from the texts. Therefore, this phase is unlikely to benefit from the use of PTMs. PTMs are most useful when used to complete repetitive tasks. Once a response has been implemented it should then be assesed to see if it has made the desired impact. 

  
\subsection{Assessment}The final stage of the POP framework is assessment. The assessment of a POP response determines 1) whether it solved the problem at hand and 2) what mechanism caused it to be effective. Assessment normally revolves around count data and statistical tools that can identify changes. This can be somewhat limiting because the count methodology is constrained by the predetermined categories that the police use to record crime. Relying on count data in this manner can cause variation in crimes within the same classification to be obscured.

Intra-crime variation might mask the success of a POP implementation. For example, a popular response to burglaries is to make targets (typically houses) more difficult to breach. Offenders may then begin to only break into the (less protected) outbuildings or to rely on open windows and doors (i.e., not forcing entry). Neither of these changes in variation would manifest in a typical count-led evaluation strategy because both still constitute burglary. Undoubtedly, however, if offenders change their techniques, then the POP response may be said to have affected them. PTMs can identify this intra-crime variation and thus supplement the count-led assessments of a POP response. It may be difficult to determine what variation one must seek. Therefore, additional consideration of the possible contexts, mechanisms, and observations is required \parencite{pawson1997realistic} to prepare the PTMs.

PTMs can enrich POP assessments considerably. A PTM can enable a more thorough assessment of intracrime variation with the same set of resources.


\subsection{PTMs in the SARA framework} The analysis above, which is based on the results from the studies that were presented in this thesis, implies that PTMs can be useful for POP practitioners. The utility of PTMs is most apparent in the initial and the final stages of the framework. In both instances, the PTMs are useful for exploring intra-crime (or intra-problem) variation. At the beginning of the application of the framework, PTMs are useful for categorising similar problems. At the end of the POP cycle, PTMs can be used to explore how criminal activity has changed even if the number of crimes that are classified in the same way remains unchanged. The utility of PTMs in the response phase is not immediately obvious. PTM usage in the assessment phase could be expanded if PTMs are improved so as to work with longer text documents.

The next section reviews the two additional questions that have a bearing on the utility of PTM usage. How is POP implemented by the police forces that hope to use PTMs? Can the labelling burden be reduced through the sharing of PTMs across police forces?

 
\section{Additional Considerations}
\subsection{POP implementation} Chapter 3 introduced two broad approaches to the implementations of POP – the generalist and the specialist approach. Under a generalist approach, individual officers are allowed to complete POP cycles. The specialist approach involves the building of specialist capacity within a police force and within the unit that conducts large-scale POP interventions. Can PTMs be used for each type of implementation? Can PTMs support both the generalist and the specialist POP approaches?

Two key factors emerge from the research that was presented on these pages. The two are problem size (i.e., the number of problems to be tackled) and technical capacity. Problem size is important because PTMs require a certain amount of sample data to learn. For example, in the PF1 burglary data, the PTMs required between 700 and 900 MOs texts to learn the classification accurately. The model could then label thousands of crimes, thus saving time. However, if the problem is small, for instance because the area of interest is not large, then the number of problem texts may be insufficient for the training and use of PTMs. Under the generalist approach, which has individual officers complete POP cycles, the number of problems may not be large. Consequently, PTMs may not be useful to such officers. Specialist teams, which might be operating on a larger scale, are more likely to face problems of an appropriate size. Therefore, the efficiencies of PTMs are more likely to be realised by specialist teams.

Secondly, and relatedly, is the resources that are required to utilise a PTM. These resources include the effort that must be expended to label the texts, the know-how that is necessary to use PTMs, and computing power. Generalist implementations may be affected by a lack of resources, whereas specifically resourced teams are more likely to be able to call on the necessary competence and hardware. Some of these issues can be overcome through more accessible tooling. Tooling can automate some of the implementation measures, and cloud-based solutions can supply additional computing power for short projects. However, these tools were not investigated in this thesis.

It is likely that, at least in the short term, PTMs will be more useful to specialised POP units that possess sufficient and appropriate data and the resources that are needed to implement them correctly.

\subsection{Model sharing} POP is traditionally associated with the development of centres of excellence and the dissemination of best practices. This approach could be transplanted to encompass fine-tuned PTMs. The results from Study 1c demonstrate that models that are trained in one police-force area can be useful in another, albeit with inferior performance. Model sharing could reduce the labelling burden that the use of PTMs entails. Even PTMs that exhibit lower performance are useful because they can be fine-tuned further on the data of the new police force and reach the desired performance level. Therefore, the transfer of models across police forces can be used as a shortcut, effectively reducing the labelling burden.

The models that are shared should target a specific problem. The problems that different police forces must solve are likely to overlap. These overlaps mean that the PTMs that are fine-tuned by one police force are likely to be useful to another police force. A certain amount of documentation must accompany the PTMs. That documentation should define the original classification task of the PTM precisely. For example, in the case of the vehicular theft classification task from Study 1a and Study 1c, the following details would need to be included in the documentation:


\begin{itemize}

\item The PTM was only fine-tuned on residential burglary data.

\item In the case of the study 1c PTM: only data from 2018/19 was used.

\item A motor vehicle included cars, vans , motorbikes and quadbikes. But not mobility scooters.

\item The vehicle, not just the car keys, had to be stolen (Some forces are also interested in the targeting of car keys).

\item The vehicle had to be removed from the property to classify as stolen.

\end{itemize}

The intricacies that emerge in the course of the labelling process as cases at the boundary of a classification are revealed indicate that there can be subtle variations in the problems on which a PTM is finetuned. If PTMs are to be shared, these subtleties must be captured and described alongside the models.

\section{Implementation Issues} This section investigate some cross-cutting issues that may prevent or delay the use of PTMs in police forces. Most of these issues were mentioned in the previous sections, but they are described here as well for completeness. The implementation issues are explored in three subsections, which concern 1) physical barriers to implementation, such as infrastructure; 2) technical barriers to implementation, which are based largely on gaps in knowledge; and 3) ethical barriers to implementation, which have less to do with the possibility of using PTMs and are more intimately connected to the desirability of their application. 

\subsection{Physical} This section is related to the physical barriers to using PTMs. These barriers emerged primarily from work and discussions with PF2. The physical barriers are generally related to the infrastructure that is required to run the PTMs. These physical barriers come in two forms, namely hardware and software.

PTMs, especially the ones that are used with longer texts, require higher-specification hardware than what is generally available to police analysts. This specialist hardware includes additional computer memory (RAM) and computing power. The required change, to meet the bare minimum requirements, is not dramatic – the costs are unlikely to exceed £1,000 per machine. In short, such an upgrade would be easy to implement if desired. 

Upgrading software is more difficult because the police use secure systems. Software is subjected to a rigorous process for preventing cyberattacks and data leaks. The work that is presented in this thesis relied heavily on open-source models and applications from the Internet. At present, they cannot be used on police computers. Although as demonstrated by this work they can be used in a secure working environment.

There are two overarching solutions to these problems – centralisation and localisation. Centralisation would involve creating a central hub of excellence where the PTMs would run. Police forces would send their data, some of which would be labelled, and their queries to the hub. The central hub would then run the PTMs, conduct explainability and bias checks, and return the results to the police forces. Localisation would entail providing individual analysts with more powerful machines and bespoke software, which is yet to be created, enabling them to run the PTMs and analyse the data. Both solutions have numerous advantages and disadvantages, some of which are explored in the technical section that follows. 
    

\subsection{Technical knowledge} Technical knowledge is related to the technical know-how that is needed to implement a solution. One of the reasons for using PTMs is that they do not require extensive knowledge. Previously, the main hurdle to utilising similar NLP techniques was the implementation of feature extraction. Since the PTMs are pretrained, feature extraction is no longer necessary. The main effort that must be expended is that of labelling the data, which requires subject-matter expertise that police forces already possess. Knowledge of other technical matters, such as hyperparameter tuning and tests for explainability and bias, can be grasped easily by a competent police analyst. Therefore, the use of PTMs in the manner in which they were employed here entails a technical burden, but police analysts are generally capable of shouldering it, especially if provided with specific training. 

Implementing PTMs can be simplified further by automating solutions in order to produce the desired results. Software applications can be built so as to abstract the intricacies of the implementation of these solutions. This abstraction requires more initial effort but would enable PTMs to be used more widely with less training. The interpretation of results, especially results on bias and explainability, would still require subject-matter knowledge, but this is a relatively light burden. As with any abstraction, a decrease in flexibility is to be expected – if a PTM performs poorly, over-reliance on automated applications may make it difficult to modify the model and/or the data.

\subsection{Ethical} There are two main categories of ethical considerations. Firstly, there are the ethical considerations that have to do with bias and explainability. They were covered in the thesis and are captured, among other issues, by the ALGO-CARE framework. The second category has to do with the data that are being analysed. The penultimate section of the data chapter discussed limitations and their implications for bias. The second part of the present section focuses on the impact of those limitations, specifically those of data coverage and information completeness.

\subsubsection{Model Implications}  The ALGO-CARE framework that was introduced in Chapter 6 allows the leadership of the police to decide whether it would be appropriate to deploy an algorithmic tool such as a PTM. The ethical implications that are derived from ALGO-CARE and which the studies explored are bias and explainability. Algorithmic tools are often biased toward certain segments of the data. This bias can manifest in the resource allocations that these tools influence. Explainability is important because it generates trust. The model should make decisions on the basis of correct information and not on the basis of spurious correlations.  

These issues were partly addressed in this thesis through the introduction of methods for the conduct of the analysis and through the presentation of results on both bias and explainability. In both respects, the results, which are limited, are promising. In particular, in the cases in which it was possible to analyse the sex and ethnicity of victims, no evidence of bias was found. The explainability results were also promising. However, there were some issues with the automatically generated text that would need to be addressed further.

The investigations that are presented in this thesis are limited. The studies only address one type of crime, one type of incident, and a limited set of classification tasks. In short, this thesis does not present a comprehensive investigation of bias in the use of PTMs. Therefore, bias remains a valid ground for ethical concerns about the use of PTMs with police data. These ethical concerns can be overcome by conducting bias checks on a case-by-case basis and by utilising the results from the PTMs only if it is established that they do not raise ethical issues.

\subsubsection{Data Implications} Chapter 3 introduced two limitations of the data that were used in this thesis. These limitations also extend to the use of PTMs for POP. The first issue is police data coverage. The police do not learn about many crimes. As explained in Chapter 3, the resultant gaps are systematic and not random. Nonrandom gaps mean that if the police only combat the crimes that they are aware of, then their resources are not applied equitably. If a biased process becomes more efficient, it is likely to exacerbate inequality further. If PTMs make POP more efficient, then POP efforts may be directed to crimes for which appropriate textual descriptions are available. The resultant pattern would be nonrandom and would likely causes POP implementations to focus on areas with more complete crime records to the detriment of others.

The second implication, which is related, concerns the completeness of the recorded crime data. If it is known that the recording of crimes is influenced by social and economic factors, then it conceivable that the comprehensiveness of the information that the police receive varies. One might also reasonably expect that the relationship between a police officer and a citizen may influence the amount of information that the latter is prepared to share with the former. Other factors, such as the absence of a common language, may also reduce the quality of crime reports. To the best of the author’s knowledge, the completeness of MO descriptions has not been researched. Likewise, the influence of victim characteristics on the completeness of police texts has not been explored. Such studies would be important because PTMs, similarly to other NLP techniques, rely exclusively on the textual descriptions that are presented to them. If those descriptions are biased in any way, then so are the results.

These considerations may affect implementation. As highlighted by the ALGO-CARE algorithm, the police are required to ensure that PTMs are used responsibly. These ethical considerations need not prevent the use of PTMs – the variable quality of police data does not obstruct other crime prevention efforts – but they must be examined in order to ensure that biases are not perpetuated.

\section{Conclusion} In summary, it has been shown that PTMs can be useful for POP practitioners. In particular, the PTMs can be used in the scanning phase to search for similar problems and in the assessment phase to understand how intra-crime variation may have changed as a result of a POP response. In each case, the PTMs are used to extract structured information from unstructured text, thus making intra-crime variation easier to quantify. In the near term, PTMs are more likely to be used by specialist POP teams because they tend to face more widespread problems and to possess the resources that are required to efficiently leverage PTMs.

There are physical, technical, and ethical barriers to the use of PTMs. The physical and technical problems can largely be overcome through the provision of additional computational resources and training. The ethical considerations are likely to prove less tractable. More research is needed to ensure that the models do not perpetuate known biases.

The next chapter concludes the thesis by indicating how PTMs can be used more broadly with police data and what other areas of research would benefit from the implementation of PTMs for POP practitioners.
