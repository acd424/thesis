\chapter{Implications for POP}


\section{Introduction} This chapter is the first chapter in the third and final part of this thesis. The aim of this part of the thesis is to draw together the lessons from the previous two parts and then asses what they mean for the future of problem-oriented policing (POP) and NLP.  This chapter will synthesise the results form the previous studies to provide answers to supporting objective 5 \say{Identify which parts of the POP process might be best supported by the use of PTMs.} . The next chapter will look more broadly at the possible future research directions for NLP with police data.

This chapter will be broken into four main sections. The first section will utilise the SARA framework for POP and  will suggest where PTMs might be best employed. The second section will review two additional considerations relating to the use of PTMs 1)how POP is implemented in a force and what affect that may have for PTM usage and 2) how forces can share fine-tuned PTMs. The final section will conclude with barriers to the implementation of PTMs in the POP cycle.


\section{Applications for POP} Part 1 of this thesis included an exploration of POP. As part of this exploration the framework for POP known as SARA was introduced, see Figure \ref{}. SARA is a four point framework, the elements of the framework are - Scanning, Analysis, Response and Assessment.  Although there is a natural order to the framework, it should be stressed that moving backwards and between the elements is encouraged to refine the process.  The following sections will briefly recap the aim of the element from the SARA framework and explore how PTMs might be applicable.

\subsection{Scan for Problems} Scanning is the first stage in the process and is centred on finding and defining the problem to be solved. As a reminder a problem is a cluster of similar related incidents, that are causing harm to the public and can be considered police business to deal with it. Problems do not have to just be crimes. In fact we have seen looking at anti-social behaviour in Study 2 that non-crimes can be serious problems. 

Generally scans are conducted with prior information about what the scan is looking for. That is the conductor of the scan already has an idea about for example the incident type and what variation they are looking for. In this instance they are trying to confirm or deny that variation and find additional problems with the same characteristics. An alternative is if the scan is searching for unknown problems for example high harm problems or new problems - but without knowing what form the problems are or what variation groups them. This alternative scan type will be returned to at the end of the section.

The scan is conducted with a general idea in mind about the problem type. As an example we can use the second problem identified in Chapter 10 (10.1.1). The detective was aware of a spate of burglaries but believed it was a result of a high proportion of outbuilding only burglaries.  In this instance the need for PTMs can be determined with ease. The first consideration is whether there exists suitable structured data to answer the question. Structured data is much easier to handle and analyse than unstructured data and so should always be the first port of call. If structured data is found then the suitability of the data should also be tested e.g. for completeness and accuracy.  For this example some structured data will be available, crimes will be classified as burglaries for instance. This structured data helps to reduce the search space but does not enable a detailed scan on the variation within the burglary. In this instance we know that the information on the variation was not accessible from structured data. Therefore PTMs will be useful to extract information from the unstructured data source and present it in a structured manner. This was achieved in this thesis by classifying burglaries as either only occurring in an outbuilding or not.  

The second consideration is the volume of potential incidents that need to be scanned. As we have seen from all studies in this thesis PTMs are a form of supervised learning and as such require labelled data for fine-tuning. For PTMs to be accurate with the data used in this thesis between 700 and 900 crimes needed to be read and labelled. This impacts the utility of the PTMs, as in order for it to be worthwhile to label the data the pool of potential incidents must be sufficiently large to make it an efficient endeavour. If the area of interest only had 100 burglaries in the last year, then PTMs may not be an efficient use of time if used for this problem alone. However if the area, and perhaps the comparison area, had thousands of burglaries then it becomes more likely that PTMs will be useful.

PTMs are likely to be useful in the scanning stage, however this will depend on a gap in knowledge from the structured data and sufficient potential problems to warrant the process costs (primarily data labelling) of using the PTMs.  This was predicated on a \emph{known} problem. There can be occasions were the exact nature of the problem is not known, as alluded to at the start of this section. In this case PTMs and NLP techniques can be used, but not as they have been used in the studies here. PTMs must be used in an unsupervised manner where the machine learning algorithm clusters the data according to variation the PTM finds, a similar method is used by \textcite{birks2020unsupervised} where they cluster burglary crimes without using any prior information on what the themes of the clusters should be. This highlights a key limitation of the use of PTMs as explored in this thesis - you must know what problem variation you are looking for in order to find it.

In summary PTMs can be useful for the scanning element of the POP process as they allow additional information to be found in unstructured information sources that can then be utilised to group problems. Once grouped the problems then need to be analysed to understand how they are occurring. This is the subject of the next section. 


\subsection{Analyse in Depth} This part of the framework is about understanding more fully how the problems are occurring. What are the underlying mechanisms that are generating the problem? Although there will be some variations between problems - what are the key areas of overlap? In this part of the framework POP practitioners must delve deeper into each problem than they did in the scanning phase. They must gain more information about the problems in order to fully understand what is happening. This more in depth analysis is likely to involve more unstructured data, and PTMs can assist with analysing this unstructured data in a systematic fashion. The relevant data sources is likely to widen, although the scanning phase may only utilise high level overviews of the problem, the analysis phase is likely to need to utilise more detailed and therefore lengthier documents. These documents could be witness statements and other police reports.

This thesis has shown that as documents get longer the ability of PTMs to efficiently analyse them becomes more difficult. The model structure of PTMs does not allow computations to scale linearly with the length of the document. This means that longer documents require more computational resources to be analysed with PTMs. Study 2 introduced a PTM (Longformer) that is designed for longer pieces of texts, although given the limited computational resources available for  the model was not fully utilised. Study 2 texts were not also not that long in terms of word count, relative to a witness statement. The police incident logs had a median of 166 words. Witness statements can run to a number of pages and with each page containing up to 400 words, it is clear that witness statements are likely to be longer than police incident logs and therefore require more computation resources. So what? Using existing PTMs for texts of this length is not well researched. However one paper from the medical literature \parencite{limitations_of_transformers} has shown that current models lose some of their effectiveness on long texts (circa 2000 words). This loss of effectiveness coupled with the high computational costs may mean that, at this stage of development, PTMs are not well suited for the more detailed work required in the analyse phase. Other NLP models may be appropriate depending on the exact nature of the problem and the texts but that is not investigated here. 

In short the analyse in depth phase is not likely to be the most appropriate phase of the SARA framework to exploit the use of PTMs.


\subsection{Respond} The respond stage is about designing and implementing a response. This phase is about considering the evidence amassed in the previous two stages and then designing a strategy to reduce the conditions for a problem to occur. Ideally the response would not lean on enforcement activity and would consider previous responses proven to work with similar problems. This stage does not entail reading similar descriptions of problems and trying to draw information out of the texts as the previous two stages did. Therefore this phase is unlikely to benefit from the use of PTMs, as the PTMs are most useful when they are conducting very similar but repetitive work.  

  
\subsection{Assessment} The final stage of the POP framework is assessment. Assessment of the POP response implemented to see if 1) it has worked in this instance and 2) to see by what mechanism it has worked. Assessment normally centres around count data and relevant statistical tools to identify a change. This can be somewhat limiting as the count methodology is constrained by the predetermined crime categories that the police use to record crime. Relying on count data like this can miss the variation in crimes within the same classification. 

This intra-crime variation might mask the success of a POP implementation. For example a popular response to counter burglaries in an area is to harden the target ( typically the house) so that it is more difficult to break into. The result of this target hardening could be a switch to only breaking into the (unhardened) outbuildings or a greater reliance on open windows and doors (i.e not forcing an opening). Neither of these changes in variation would show up in a typical count led evaluation strategy, as they both still constitute a burglary. But undoubtedly the POP response has had an effect on the offenders. PTMs, as has been demonstrated in this thesis, can find this intra-crime variation and so can be useful to supplement the count led assessments following a POP response. Knowing what variation to look for though may not be obvious, and will therefore require additional thoughts about the possible contexts, mechanisms and observations required \parencite{pawson1997realistic} to prepare the correct PTMs to find the possible variations.

PTMs therefore could potentially provide utility by making the POP assessment much richer. PTMs can allow a more thorough assessment of intra-crime variation than would previously been possible with the available resources. 


\subsection{PTMs in the SARA framework} The above analysis, based on the results from the studies conducted in this thesis, concludes that PTMs can provide utility for POP practitioners. Particularly this utility will come at the beginning and end of the framework. In both instances the PTMs are useful to explore the intra-crime ( or intra-problem) variation. At the beginning of the framework  PTMs will be useful for categorising similar problems. At the end of the POP cycle PTMs can be used to explore how crime activity has changed, or not, even if the number of crimes in the same classification remains unchanged. Utility for PTMs in the respond phase is not immediately obvious. PTM usage in the Assessment phase could be expanded if PTMs are developed to work with longer text documents. 

The next section will review two additional considerations that can affect the utility of PTM usage. Firstly how POP is implemented in the police force hoping to use the PTMs and secondly reducing the labelling burden through PTM sharing.

 
\section{Additional Considerations}
\subsection{POP implementation} Chapter 3 introduced two broad implementations of POP - the generalist and the specialist approach. The generalist approach allows individual officers to conduct POP cycles. The specialist approach involves the building of specialist capacity within a police force and the unit conducting larger POP interventions. The question that this presents is can PTMs be used for each implementation type? Can PTMs support both the generalist and the specialist POP approaches? 

In considering the question of PTM support to the generalist and specialist POP approaches two dominating factors emerge from the research conducted here. They are problem size ( i.e the number of problems to be tackled) and technical capacity. 

Firstly problem size, the number of problems in the area of interest. Problem size is important as PTMs require a certain amount of sample data for them to learn. For example in the PF1 Burglary data the PTMs required 700 to 900 MOs texts to learn the classification accurately, the built model was then able to go onto label thousands of crimes and thus time was saved. However if the problem size is small, for instance because the area of interest is small, then there may not be enough problem texts for PTMs to be trained and used.  In the generalist approach where individual officers are conducting their own POP the number of  problems may not be that large, and so consequently PTMs may not be that useful to them. Specialist teams who might be operating on a larger scale are more likely to have a more appropriate problem size, and so the efficiencies of PTMs are likely to be leveraged.  

Secondly, and relatedly, is the resources required to utilise the PTM. This resource includes the effort to label the texts, the know how to use the PTMs and the computing power to run the PTMs. Generalist implementations may not have the resources at hand to use the PTMs effectively, whereas specifically resourced teams are more likely to be able to call on the skills and hardware required to operate the PTMs. Some of these issues can be overcome through more accessible tooling for the PTMs. Tooling that can automate some of the implementation requirements and cloud based solutions are able to more easily offer additional computing power for short projects. However these potential tools were not investigated in this thesis.

It is therefore more likely that, at least in the short term, that PTMs will have more utility with specialised POP units who have sufficient appropriate data and the resources to implement the PTMs correctly.

\subsection{Model sharing} POP has a history of developing centres of excellence and sharing best practice. It is possible that this approach could be spread to fine-tuned PTMs. The results of Study 1c demonstrated that models trained in one police force area have utility in a second police force area, albeit with reduced performance. This means that model sharing between police forces could be a method to reduce the labelling burden for the use of PTMs. Even PTMs with reduced performance are useful as they can be further fine-tuned on the new police force data to reach the desired performance.  Therefore transferred models across police forces can be used to to short cut the fine-tuning process, effectively reducing the labelling burden. 

Model shared will be for a specific problem, however it is likely that there will be problem overlaps between different police forces. Problem overlaps mean that the PTMs fine-tuned in one police force are likely to be useful in a second police force. However, in order to share PTMs effectively there will have a certain amount of documentation that accompanies the PTM to help define precisely what classification task the PTM was used upon. For example in the case of the motor vehicle theft classification task implemented in studies 1a and 1c the following kinds of detail would need to be added into the documentation to specify the PTM:

\begin{itemize}

\item The PTM was only fine-tuned on Residential Burglary data.

\item In the case of the study 1c PTM: only data from 2018/19 was used.

\item A motor vehicle included cars, vans , motorbikes and quadbikes. But not mobility scooters.

\item The vehicle, not just the car keys, had to be stolen (Some forces are also interested in the targeting of car keys).

\item The vehicle had to be removed from the property to classify as stolen.

\end{itemize}

The intricacies of each fine-tune, often developed through the labelling process as the edge cases become apparent, show that there can be subtle variations in what problem the PTM is fine-tuned on. If PTMs are to be shared, these subtleties will need to be captured and shared alongside the models.

\section{Implementation Issues} This section investigate some cross-cutting issues that may prevent, or slow, the use PTMs in police forces. Most of these issues have been touched on in the previous sections, but are captured here in one section for completeness. The implementation issues are explored under three subsections: 1) Physical - physical barriers to implementation such as infrastructure 2) Technical - based largely on gaps in knowledge and 3) Ethical - which is less about \emph{can} these PTMs be used and more about \emph{should} they be used.

\subsection{Physical} This section relates to the physical barriers or utilising PTMs. These barriers are drawn primarily through working and discussions with PF2, although this is limited they are generally representative of police forces in the UK. The physical barriers generally relate to the infrastructure required to run the PTMs. These physical barriers are two-fold, hardware and software. 

Firstly hardware. PTMs, especially for the longer texts require higher specification hardware than is generally available to police analysts. This specialist hardware includes extra computer memory (RAM) and compute power. The required jump from what the police analysts have now to what they need is not vast, it perhaps represents additional costs in the order of hundreds of pounds per machine rather than thousands. In short it is easily achievable if desired.  

Software is harder, as the police use secure systems that require software to go through a rigorous process to guard against cyber attack and data leaks. The work in this thesis relied heavily on drawing open source models and applications from the internet that is not currently possible on police computers. 

There are two possible over-aching solutions to these problems 1) centralisation 2) localisation. Centralisation would involve creating a central hub of excellence where the PTMs would be ran. Police forces would send their data (some of which would be labelled) and their queries to the hub. The central hub would then run the PTMs, including explainability and bias checks returning the results to the police force. Localisation would empower individual analysts with more powerful machines and bespoke software (yet to be created) that would allow them to run the PTMs and analyse the data for themselves. Both solutions have further advantages and disadvantages, some of which are explored next in the technical section. Down selecting to one solution though is not important here to demonstrate that physical issues can be overcome.    

\subsection{Technical knowledge} Technical knowledge relates to the technical know-how to implement the solution. One of the reasons for selecting PTMs was that they can be used without extensive knowledge. Previously the main hurdle for utilising similar NLP techniques was implementing feature extraction, however as the PTMs have been pre-trained there is no requirement to do that. The main  resource effort remaining is labelling the data, which requires subject matter expertise already found in police forces. Other technical knowledge such as hyper-parameter tuning and investigating explainability and bias  once the method is established are easily within the grasp of a competent police analyst. Therefore there is a technical burden to use PTMs as they have been used in this research, however police forces generally have analysts capable to conduct the work they may just require training in the specifics of how to implement the selected solutions.  

Implementing PTMs can be further simplified by automating the solutions to produce the desired results. That is software applications can be built to abstract away the some of the intricacies of implementing these solutions. This abstraction will require more initial effort but will allow wider usage of PTMs with less training. The interpretation of the results, especially the bias and explainability results,  will still require subject matter knowledge, but this is a relatively lower burden. As with any abstraction though there is a decrease in flexibility for the solution and so if a PTM has poor performance the ability to modify the model and or the data may be reduced if there is an over reliance on an automated application.  

\subsection{Ethical} Ethical consideration fall in to two main sections. Firstly there are the ethical considerations around bias and explainability that have been captured in the thesis so far and are captured, amongst other issues by the ALGO-CARE framework. The second aspect of the ethical implementation issues relate to the data being analysed. In the data chapter the penultimate section discussed the limitations of the data and how that may introduce bias. The second part of this ethical section focuses on the impact of those limitations, specifically the first two limitations - police data coverage and information completeness.

\subsubsection{Model Implications}  The ALGO-CARE framework introduced in Chapter 6 is a guiding framework to allow police leadership to decide if it is appropriate or not to deploy an algorithmic tool, such as a PTM. The ethical implications, derived from ALGO-CARE that the studies explored were bias and explainability. Bias is important because algorithmic tools are often biased in performance to certain sections of the data, and this bias has the potential to perpetuate biases into the application of resource that these tools influence. Explainability is important as it helps to generate trust in the model. Trust that the model is making decisions upon the correct information presented to it and not spurious correlations.  

These issues were partly addressed in this thesis, through introducing methods to conduct the analysis and present the results of both bias and explainability. In both respects the limited set or results that were derived were promising. In particular where victim sex and ethnicity characteristics could be analysed no issues of bias were found. The explainability results were also on the whole promising, although there were some issues with automatically generated text that would need to be further addressed.

However the extent of the investigations in this study were limited. The studies only addressed one crime type, one incident type and a limited set of classification tasks. In short this thesis does not present a comprehensive investigation into bias for the use of PTMs, and therefore these model implications are still valid impediments to utilising PTMs with police data. Though these model ethical concerns can most easily be addressed by analysing the police data with the PTMs, then conducting bias and explainability checks before utilising the results from the PTMs if there are no ethical issues. 

\subsubsection{Data Implications} In Chapter 3 two limitations on the data that was used for this thesis were introduced. These limitations also extend to the use of PTMs for POP. The first issue is police data coverage, that is of all the total crimes committed how many do the police know about. As explained in Chapter 3 the main issues with the coverage of police data is that the gaps are systematic - they are not random. Non-random crime knowledge gaps mean that if the police apply effort against only crimes they are aware of then their resources are not applied in an equitable way. This limitation is well understood, but becoming more efficient at a biased process is likely to further exacerbate issues of inequality. Reflecting on using PTMs for POP, this could mean that if PTMs make POP more efficient then POP effort may be applied to where good textual descriptions of crimes are recorded, this would follow a non-random pattern and would likely bias POP implementations to areas where crime recording is more complete.

The second, but related implication, is the completeness of the crime data that is recorded. Here I refer to only the completeness of the text description to describe the crime. If we know that crime recording of crimes is influenced by social and economic factors, then it is a possibility that if a crime is recorded that the level of detail that can or will be given to the police may vary. We might also reasonably expect that the relationship between the police officer and the citizen may influence the amount of information that is passed about a crime. Other factors such as the absence of a common language may also degrade the quality of the details of a crime that are recorded. To my knowledge the completeness of MO descriptions has not been tested, nor has the influence of victim characteristics on the completeness of texts been explored. This is an important research area as PTMs, and other NLP techniques, can only rely on the textual descriptions presented to them - and if they are biased in anyway then that will in turn bias the results.

These two data implications may have an impact on implementation. As highlighted in the ALGO-CARE algorithm police have a duty of care to ensure that PTM implementations are used in a responsible way. These ethical considerations need not prevent the use of PTMs, just as the quality of police data does not stop other crime prevention efforts,  but they must be guarded against to ensure that the biases already present are not further engrained.  

\section{Conclusion} In summary it has been shown that PTMs can be useful for POP practitioners. In particular the PTMs can be used in the Scan phase to search for similar problems and in the assessment phase to understand how intra-crime variation may have changed in as a result of the POP response. In each case the PTMs are used to extract structured information from unstructured text, therefore making the intra-crime variation easier to quantify. In the near term PTMs are more likely to be usable by specialist POP teams as they tend to have the more widespread problems and resources that are required to leverage the PTMS most effectively. 

However there are barriers to utilising PTMs. Outlined here as physical, technical and ethical barriers. The physical and technical problems can largely be overcome through additional computational resources and training. The ethical considerations are likely to require more thought as they will require more research to ensure that the models are not unduly perpetuating known biases. 

The next chapter builds on this Chapter by investigating more broadly how PTMs can be used with police data and what other areas of research would benefit the implementation of PTMs for POP practitioners. 

