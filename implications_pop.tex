\chapter{Implications for POP}


\section{Introduction} This chapter is the first chapter in the third and final part of this thesis. The aim of this part of the thesis is to draw together the lessons from the previous two parts and then asses what they mean for the future of POP and NLP.  This chapter will synthesise the results form the previous studies and describe what implications they have for Problem-oriented policing(POP). The following chapter will look more broadly at the possible future research directions for NLP with police data.

This chapter will be broken into four main sections. The first section will review the results from the studies and draw conclusions based on the main research question and the supporting objectives. The next section will explore the limitations fo this work. The third section will utilise the conclusions and limitations presented and using the SARA framework for POP will suggest where PTMs might be best employed. The final section will conclude with barriers to the implementation of PTMs in the POP cycle.

As a reminder the research question was:

\textbf{Can PTMs be used efficiently to extract information from police free-text data, and if so what practical applications for problem-oriented policing does this approach have?}

The supporting objectives were:
\begin{enumerate}
\item {\bf Identify the extent of NLP usage with police data.} This will be largely conducted in the literature survey which will be the focus of Chapter 6.

\item {\bf Evaluate how effective PTMs are with MO data.} PTMs will be formally introduced and explained in Chapter 5. MO data will be introduced in Chapter 8. Study 1 will investigate the use of PTMs to classify MO text data.

\item {\bf Evaluate how effective PTMs are with Police Incident data.} As mentioned police incident data is another source of information on problems the police face. Using PTMs to classify police incident logs will be investigated in Study 2 

\item {\bf Evaluate how effective Active Learning is with police data.}  Active Learning is a method to reduce the amount of data that PTMs need to learn. It has been found to work with other data types but its effectiveness with police data is unknown. Active Learning is introduced in Chapter 4 and studied in study 1.

\item {\bf Identify which parts of the POP process might be best supported by the use of PTMs.} The POP process will be explained fully in Chapter 3. It is likely that different parts of the process will find differing uses and utility for PTMs. Lesson for POP will be drawn from both studies and outlined in Part 3. 

\item {\bf Identify implementation barriers for PTMs.} Any new process is likely to have implementation barriers, these are important to identify so that they can be minimised. Discussed in Part 3. 

\end{enumerate}

\section{Synthesis and Assessment of Findings from Case Studies.} This section will use the supporting objectives as a handrail to explore the cross cutting issues identified in the studies as outlined in part 2 of this thesis.

\subsection{Extent of NLP usage with police data.} The literature survey in Chapter 6 covered this. The findings were that there was no recorded use of PTMs with police free text data. This was 



\section{Limitations}


\section{Applications for POP}

S
A
R
A

\section{implementation issues}

physical

technical

Ethical


